\documentclass[a4paper,11pt,twoside,openright]{scrbook}

\usepackage{../jnwthesis}
\usepackage{amsmath}
\usepackage{lipsum}
\usepackage{standalone}
\standalonetrue

\bibliography{/Users/jonwells/Documents/bibtex/Thesis}
\graphicspath{{../figs/}}


\begin{document}

\chapter{Conclusion}\label{chapter:conclusion}
\section{Insights into the nature of protein complexes}
In this work I have attempted to develop our understanding of the mechanisms by which the cell regulates protein complex assembly, and of the implications this process has for the proteome as a whole. After a literature review covering the currently available methods that can be used to study protein complexes, I began by describing a study of gene order in prokaryotic operons, demonstrating that bacterial gene order matches the assembly order of protein complexes. This was followed by work on protein degradation kinetics in mammalian cells, which revealed novel insights into the nature of eukaryotic protein complex assembly. Using a simple model developed in that chapter, I then showed how rapid degradation of excess heteromeric subunits could explain the phenomenon of protein attenuation in aneuploid cells. Finally, I described the Hawk proteins - an important family of condensin and cohesin regulators descended from an ancestral HEAT repeat protein.

Aside from this last excursion into evolution, the primary focus of this thesis has been on regulatory mechanisms governing the assembly of protein complexes, both in prokaryotes and eukaryotes. Chapter \ref{chapter:operons} focused on prokaryotes - bacteria in particular - demonstrating that gene order in operons is under evolutionary selection pressure to match the assembly order of heteromeric protein complexes. Earlier work had shown that gene fusions were more likely to be fixed if they preserved the order of assembly \cite{Marsh2013}, but since these are relatively rare events, the finding that operon gene order is also constrained provides stronger evidence for the biological importance of ordered assembly. It is now clear that significant fitness benefits can be achieved via mechanisms that guide assembly down thermodynamically favourable pathways.

A second important implication from this study relates to the location of assembly in bacteria. In order for gene order to have a beneficial effect on assembly efficiency, assembly must be taking place very close to the site of translation. This implication has been directly supported by an experimental paper in which showed that operon-encoded luciferase subunits associated co-translationally in \textit{E. coli} \cite{Shieh2015a}. Furthermore, numerous empirical reports and theoretical arguments in the literature indicate that this is not a phenomena unique to prokaryotic operons \cite{Wells2015,Natan2017}.

Eukaryotic protein complexes do not benefit from being encoded in operons, and therefore other mechanisms must have evolved that allow robust assembly of complexes. An important distinction between prokaryotes and eukaryotes is that the latter typically have smaller effective population sizes - this places constraints on the ability of selection to leverage small fitness benefits \cite{Kimura1962,Lynch2011}. However, protein complexes are such a fundamental part of the cell that the process of assembly is certainly not left entirely to chance. One such mechanism with the power to facilitate assembly of eukaryotic complexes is pointed to in chapter \ref{chapter:degradation} - a study of protein degradation kinetics. This study revealed that many proteins are degraded non-exponentially, and this appears to be a result of rapid degradation of excess protein complex subunits.

The key finding from this work is that eukaryotic protein complexes - in contrast to prokaryotic ones \cite{Li2014b} - are not expressed at levels in accordance with their stoichiometries. Instead, eukaryotic protein complex assembly appear to assemble around core subunits that are overexpressed relative to their stoichiometric requirements, with excess protein being rapidly degraded (figure \ref{figure:conclusionmodel}). These are fundamentally different mechanisms, and reflect the differences in genome architecture between these domains of life.

\begin{figure}
    \includegraphics{c6_fig_model}
    \caption[Models of prokaryotic and eukaryotic heteromer assembly]{\sffamily \textbf{Models of prokaryotic and eukaryotic heteromer assembly} \\ \small (A) Thermodynamically favourable assembly pathway for a hypothetical protein complex. (B) Prokaryotes frequently encode heteromeric protein complexes within operons. Subunits from these are encoded in the same order as they assemble, and are expressed in proportions closely matching the stoichiometry of the complex. Furthermore, they benefit from a high degree of colocalisation afforded by operons, and cotranslational assembly is almost certainly commonplace. (C) Eukaryotes do not encode protein complexes within operons, but instead appear to compensate for the lack of co-regulation incurred by this by overexpressing core subunits and rapidly degrading the excess.}
    \label{figure:conclusionmodel}
\end{figure}

A consequence of the behaviour of eukaryotic heteromers during assembly can be seen in aneuploid cells. Specifically, in the attenuation of proteins that occurs in response to increasing chromosomal copy number. From collaborative work carried out with others \cite{McShane2016} and my own later investigations on the topic (chapter \ref{chapter:aneuploidy}), it now seems clear that this phenomenon is a direct result of degradation of excess protein subunits, as has been suggested by others \cite{Dephoure2014,Mueller2015,Goncalves2017}. Having said that, there are still questions surrounding the degree to which protein aggregation occurs in aneuploid cells, and how much this confounds observations of attenuation.

Finally, whilst the characterisation of the Hawk family of condensin and cohesin regulators does not directly relate to assembly, it does shed light on the evolution of a fascinating group of protein complexes, namely the SMC-kleisins. The fact that the hawks appear to originate very early in eukaryotic history suggests that they played an important role in the evolution of condensin and cohesin's present-day functions, and highlights more broadly the importance of subunit gain and loss in the evolution of protein complexes \cite{Seidl2009,Wan2015}. Enticingly, the relationship between the hawks and the clathrin adaptor proteins hints at a link with the proto-coatomer hypothesis \cite{Devos2006,Field2011}, and raises questions about how closely intertwined the emergence of linear chromosome condensation was with the evolution of the nucleus.

\section{Questions arising from this work}
Many new - and some old - questions are suggested by this work. As an obvious starting point, whilst coexpression of protein complex subunits is ubiquitous, it is still not at all clear how this is achieved in eukaryotes. Although eukaryotic gene order is non-random (see review by Hurst et al. \cite{Hurst2004}), there is very little clustering of genes in the manner of operons, which might lead to co-regulation due to spatial proximity of active transcription factors. However, even in the absence of spatial clustering, it seems to be rare for protein complexes to be co-regulated by single transcription factors \cite{Tan2007}.

This is a question that, whilst especially pertinent to the assembly of protein complexes, is relevant to any biological processes that require coordinated gene expression. The failure of simple mechanisms to explain coexpression suggests that it is an emergent property of the transcription factor network, involving multiple transcription factors operating at different regulatory levels \cite{Tsai2007,Muhammad2017}, but explaining this will therefore require a deeper understanding of cellular transcription factor networks. Projects such as ENCODE have done much to further progress in this domain, but there are still fiery debates in the field, even about seemingly basic issues such as the size of the functional genome \cite{Neph2012,Graur2013,Graur2017}.

One point of interest from the work on gene order in operons was the implication that cotranslational assembly is likely the default for operon-encoded protein complexes. In the case of prokaryotic operons, it is easy to imagine this being true, but less so for eukaryotes, in which each protein is encoded on a separate mRNA. Nonetheless, there is considerable evidence for cotranslational assembly taking place in eukaryotes too, not only for homomeric complexes but also for heteromers - two reviews on this topic are included in appendix \ref{appendix:published}. Still, there are plenty of unanswered questions on the topic of where in the cell assembly takes place. Without doubt, extensive subcellular localisation of proteins does occur, but whether or not this is sufficiently specific as to make an appreciable difference on assembly efficiency is still unknown.

With respect to the work on protein degradation kinetics, there are several leads that would be interesting to follow up on. The fact that both NED character and attenuation in aneuploid cells are reduced by repressing the proteasome motivates one to ask what role ubiquitination plays in the regulation of protein complex assembly and degradation? There is evidence that both ubiquitination and acetylation sites are enriched in heteromeric protein complexes \cite{Chen2014,Choudhary2009}, which hints at considerable complexity in the regulation of protein complex degradation \cite{Caron2005}. An interesting project would be to ask whether or not such modifications affect the ability of protein complexes to assemble - one way of investigating this might be to look at whether there is any enrichment or depletion of modified lysine residues in the interface regions of protein subunits.

Also, whilst it seems likely that rapid degradation of excess subunits is of benefit partly because it reduces the likelihood of off-target interactions  - the need for which is an important constraint on protein evolution \cite{Yang2012} - there are more subtle mechanisms of controlling expression that could also be facilitated by this system. Under the model described in chapter \ref{chapter:degradation}, the simplest interpretation of a protein complex containing both ED and NED proteins is that the NED proteins appear as such because they are present in excess, and thus have both bound and unbound subunits, whereas the ED proteins are fully bound, hence the single degradation rate.

This implies that the ED subunits are limiting for the overall abundance of the complex, and therefore by controlling the expression of a small number of ED subunits, overall levels of the protein complex could be modulated relatively simply. Testing this hypothesis would be challenging, though not impossible, and would require changes in subunit expression to be monitored over extended time periods or in response to cell perturbations. Interestingly, some circumstantial evidence for this idea exists in a nice paper showing that many protein complexes involved in the yeast cell cycle consist of a mixture of constitutively and periodically expressed subunits \cite{Lichtenberg2005}.

There is still scope for further investigation into the attenuation of proteins in aneuploid cells. For example, although there seems to be a strong connection between NED and attenuation, the latter is a phenomena caused by fairly drastic changes to the karyotype of a cell. Duplicating a chromosome could reasonably be expected to increase protein aggregation in the cell, and from preliminary (unpublished) work this certainly seems to be the case, with members of protein complexes being particularly affected - the question now is to try and understand what dictates why some proteins aggregate whilst others are degraded.

\section{Closing remarks}
Over the preceding pages I have presented work relating to numerous aspects of protein complexes and the mechanisms by which they assemble. Beyond these contributions, there is still an enormous amount to be discovered, and it is my hope that some of the value of this work will be in the new research it motivates. With the wealth of additional biological data that becomes available each year, there is plenty of opportunity to learn more, and if the many excellent papers I have read over the last three years are anything to go by, then protein complexes will continue to be a fascinating area of study for many years to come.

\end{document}
