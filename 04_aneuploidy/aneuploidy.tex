\documentclass[a4paper,11pt,twoside,openright]{scrbook}

\usepackage{../jnwthesis}
\usepackage{amsmath}
\usepackage{lipsum}
\usepackage{standalone}
\standalonetrue

\bibliography{/Users/jonwells/Documents/bibtex/Thesis}
\graphicspath{{../figs/}}


\begin{document}

\chapter{Autosomal dosage compensation in aneuploid cells}\label{chapter:aneuploidy}

\section{Introduction}

Cancer cells are renowned for their high levels of chromosomal instability and
unusual, sometimes bizarre karyotypes. The state in which a cell has an abnormal
number of chromosomes is known as aneuploidy, and occurs to degrees in all known
cancers. However, it is common even in non-cancer cells, and is a pervasive
feature in most eukaryotic organisms. Isolates of wild yeast strains for example
have been found to harbour a variety of different karyotypes, \cite{Hose2015}.
In humans, approximately 0.1\% of the population carries an extra copy of
chromosome 21 \cite{Presson2013}, a karyotype that famously results in Down's
syndrome. In most cases however, aneuploidies that have been acquired through
the germ-line or early in development are lethal - highlighting this, a recent
study of spontaneous miscarriages found that approximately 45\% were the result
of aneuploidies \cite{Jia2015}, and the true figure is probably higher still,
since studies in mice suggest that mosaic aneuploid embryos fail to develop much
beyond gastrulation, and thus would often pass clinically undetected
\cite{Lightfoot2006}.

Mechanistically, aneuploidy is the result of chromosomes failing to separate
properly during cell division, either through non-disjunction of sister
chromatids or delays in the movement of chromosomes to opposite poles of the
cell during anaphase \cite{Compton2011}. For a cell that gains a chromosome, the
immediate effect is a doubling of the copy number of all the genes residing on
it, thus leading to a significant increase in mRNA and protein production.
However, a recurring feature noted in several studies is that a considerable
number of these proteins are attenuated compared to their expected abundances.
\cite{Stingele2012,Dephoure2014,Goncalves2017}.

This attenuation is generally thought to be caused by post-translational
degradation of excess protein complex subunits, since attenuated proteins are
enriched in protein complexes, and their mRNA transcript abundances scale
correctly with copy number. Strongly supporting this, the model presented in the
previous chapter explaining non-exponential degradation of proteins also
correctly predicts the attenuation of proteins in aneuploid cells. In this
chapter, I will briefly elaborate on this idea and attempt to explain some of
the similarities and differences between non-exponential degradation in wild
type cells and protein attenuation in aneuploid cells.

\clearpage

\section{Results}
\subsection{Attenuation of protein complex subunits is unique to heteromers}
First, to confirm that attenuation of protein complex subunits is a feature
unique to heteromers, I made use of data from Dephoure et al.
\cite{Dephoure2014}. This dataset describes the relative fold change in the
abundance of 2,581 genes from disomic \textit{S. cerevisiae} strains. Using the
same definition given by Dephoure et al., attenuated proteins are those
satisfying:
\begin{displaymath}
    \log_{2} \left( \frac{disomic}{wildtype} \right) \leq 0.6
\end{displaymath}
where $disomic$ and $wildtype$ refer to the abundance of individual proteins in
the two different cell populations. This value ensures that the observed SILAC
ratio is at least three standard deviations away from the expected mean value of
1.0 (equivalent to doubling abundance). One of the key findings from this paper
was that attenuated proteins were enriched in a set of multi-subunit complexes
obtained from a dataset of yeast complexes compiled from the experimental
literature \cite{Pu2009}. After replicating this finding using the same dataset
(figure \ref{suppfigure:aneuploidy_qstype}), I mapped this aneuploidy dataset
onto structures from the PDB. This structural dataset confirmed these earlier
observations that attenuation of expression is more common for members of
multi-subunit protein complexes, and demonstrated that attenuation was
restricted to heteromeric subunits (figure \ref{figure:aneuploidy_qstype}).

\begin{figure}[h]
    \makebox[\textwidth]{\includegraphics[width=\textwidth]{c4_fig_qsdsratio}}
    \caption[Attenuation of protein complexes is unique to heteromers]{\sffamily
    \textbf{Attenuation of protein complexes is unique to heteromers} \\ \small
    Members of protein complexes are significantly more likely to be attenuated
    upon doubling of gene copy number. However, this effect is almost entirely
    driven by heteromeric complexes, with disomic ratios that differ
    significantly from that seen in monomers (median log2 disomic ratios of
    0.623 and 0.948 respectively, Wilcoxon rank sum test p-value = 4.315e-22).
    In contrast, homomers (median 0.942) behave much the same as monomers and
    the difference between them is not at all significant.}
    \label{figure:aneuploidy_qstype}
\end{figure}

Importantly, this observation is consistent with the model presented in chapter
4, in which non-exponential degradation is a consequence of excess protein
complex subunits being degraded more rapidly than bound ones. Adapted to this
situation, disomic ratio is dependent on the ratio of bound to unbound subunits.
Duplicating a single subunit in a heteromer increases the unbound, unstable
fraction of that protein, which is promptly degraded, leading to a lower disomic
ratio. Proteins that are predominantly monomeric should not experience any
systematic attenuation, since their degradation rate will not be affected by
binding partners and should remain roughly constant. Likewise, subunits of
homomeric complexes will not be attenuated since changes in gene copy number
will not cause stoichiometric imbalances.

\subsection{Similarities and differences between wild type subunit degradation
and aneuploid attenuation}
Many of the features we see in ED and NED proteins reappear here, suggesting
that they are different manifestations of the same underlying biological
phenomena. For example, as complex size increases, the average disomic ratio of
subunits decreases (figure \ref{figure:aneuploidusubs}), analogous to the
enrichment of NED proteins in large complexes. As before, this is probably
explained by the fact that larger proteins have a greater proportion of obligate
subunits that are protected from degradation once bound.

\begin{figure}[h]
    \makebox[\textwidth]{
        \includegraphics[width=\textwidth]{c4_fig_usubsdsratio}
        }
    \caption[Degree of attenuation increases with increasing complex
    size]{\sffamily \textbf{Degree of attenuation increases with increasing
    complex size} \\ \small The abundance of monomers and homomeric subunits is
    largely determined by gene copy number, with log\textsubscript{2} disomic
    ratios being approximately normally distributed about 1. Heteromeric
    subunits however become increasingly likely to be attenuated as the
    increasing number of unique subunits. “Significant” attenuation is defined
    here by a threshold value of < 0.6, highlighted by the dashed red line.
    Structural and non-structural datasets have been combined here.}
    \label{figure:aneuploidusubs}
\end{figure}

In contrast, if one imagines a typical heterodimer, the observed attenuation for
each subunit will be determined by the degree to which its binding partner
buffers that subunit. This in turn is dependent on binding affinities,
dissociation rates and starting concentrations. For example, if the starting
concentration of A is much higher than B, then even doubling [B] will have
little effect on the amount of unbound B, and thus it will appear to be
non-attenuated.

One observation raised in Dephoure et al. is that almost all complexes have at
least one non-attenuated subunit. An interesting idea mooted to explain this is
the possibility that complexes require at least some stable subunits to act as a
scaffold for the rest of the complex. If this were the case, then we would
expect to see non-attenuated proteins being amongst the first to assemble.
However, if we compare the disomic ratio of those subunits that assemble first
against those that assemble last (figure \ref{figure:aneuploidy_assembly}), we
find a clear (albeit weakly significant when controlling for number of unique
subunits) tendency for early-assembling subunits to be attenuated. This finding
matches the observations showing that NED subunits tend to assemble earlier than
ED subunits, and is incompatible with the idea of stable proteins acting as
scaffolds for protein complex assembly. Consistent with this, attenuated
proteins also tend to form larger interfaces.

\begin{figure}[h]
    \makebox[\textwidth]{\includegraphics[width=\textwidth]{c4_fig_assembly}}
    \caption[Subunits that bind late to the complex are less likely to be
    attenuated]{\sffamily \textbf{Subunits that bind late to the complex are
    less likely to be attenuated} \\ \small Red line indicates `attenuated'
    threshold. P-values are Wilcoxon rank sum tests indicating the difference in
    disomic ratio between the first and last subunits to assemble. `Mid'
    subunits are all those that are neither first nor last to assemble. The
    non-significant values are probably mostly due to the fact that the `first'
    and `last' subunit numbers are highly limited by the number of available PDB
    structures, which is relatively low for the larger complexes.}
    \label{figure:aneuploidy_assembly}
\end{figure}

An interesting difference between NED proteins and attenuated proteins is in
their abundance. At first glance, one might expect to see a similar case as for
the NED proteins, where within complexes NED proteins tend to be more abundant
than ED. However, when we look at this for the disomic ratio data there is no
significant difference between proteins whose abundance is attenuated and vice
versa (figure \ref{suppfigure:aneuploidy_abundance}). On reflection, this makes
sense - in wild type cells, the more abundant protein will be degraded
non-exponentially, whereas in aneuploid cells, the natural abundance of the
protein is rendered insignificant compared to the effect of copy number changes.

Less easy to explain however is the fact that attenuated proteins appear to show
a greater propensity to disorder than non-attenuated (figure
\ref{figure:aneuploidy_disorder}). This is in contrast to NED and ED, in which
the latter were found to be more disordered \cite{McShane2016}. One possibility
that could explain this discrepancy is that in aneuploid cells - in which
protein abundance has been drastically increased - disordered proteins are
simply aggregating instead of being degraded. Though I have not yet tested this
idea formally, there is evidence to suggest that disorder is correlated with
aggregation propensity \cite{Carvalho2013}.

\begin{figure}[h] \fcapsideright {\caption[Attenuated proteins show increased
    disorder]{\sffamily\textbf{Attenuated proteins show increased
    disorder}\newline \small A single disorder score for each protein in a set
    of protein complexes was produced by taking the median value from all
    residues, where per-residue disorder scores were predicted using IUPred
    \cite{Dosztanyi2005}. Proteins were taken from a set of complexes produced
    by combining structural data with that from the Pu et al. dataset
    \cite{Pu2009}. P-value calculated with Wilcoxon rank-sum
    test.}\label{figure:aneuploidy_disorder}}
{\includegraphics[width=0.5\textwidth]{c4_fig_disorder}} \end{figure}

\subsection{Aneuploidy leads to increased heteromeric protein aggregation}
Protein aggregation is in an important aspect of aneuploidy that has not yet
been explored in a high-throughput manner. Beyond the possibility that it could
explain this discrepancy between disorder propensity in NED and attenuated
proteins, it is also possible that the apparent attenuation of some proteins
could be due instead to aggregation. However, measuring this is technically
challenging due to the difficulty of isolating and quantifying the aggregated
fraction of cells. Nonetheless, a pilot study carried out in collaboration with
the Amon lab at MIT has produced some interesting results that merit further
exploration in the near future.

The protocol followed for this pilot is essentially the same as that described
in Dephoure et al., with the key difference being that instead of carrying out
measurements on the whole cell lysate (in 8M urea), the lysate is spun down and
the aggregate fraction extracted for further use. Disomic:wild type ratios were
calculated in the same manner as before, with proteins being classified as
either `aggregating' or `highly aggregating'. The threshold for the latter is
set semi-arbitrarily at a disomic ratio $\geq 1.5$, where a value of one would
indicate a straightforward doubling of protein abundance in the aggregate
fraction.

At least in aneuploid cells in which chromosome copy numbers have been
increased, one would expect to see an increased propensity for aggregation
across the cell, and particularly for proteins on affected chromosomes. This is
certainly the case, and furthermore, there is an increased tendency for
heteromeric proteins to aggregate more than expected upon copy number
duplication (odds ratio = 2.392818, p-value = 0.038, comparing heteromeric to
monomeric proteins). As was the case for attenuation in whole-cell measurements,
there was no significant difference between monomeric and homomeric proteins.

I then compared those proteins which were found to be highly aggregating with
those that are attenuated and found a weak tendency for the different classes to
be mutually exclusive (figure \ref{figure:aneuploidy_comparison}). This suggests
that proteins are either aggregate extensively, are actively attenuated, or
behave neutrally and simply double their abundance in either soluble or
insoluble fractions of the cell. Importantly, this tells us that attenuation in
aneuploidy is unlikely to be an experimental artefact caused by the aggregated
fraction of proteins not being efficiently picked up in whole-cell lysates.

\begin{figure}[h] \fcapsideright {\caption[Comparison of attenuation and
    aggregation propensity]{\sffamily\textbf{Comparison of attenuation and
    aggregation propensity}\newline \small Separating proteins into groups
    according to whether they are attenuated or highly-aggregating demonstrates
    that for proteins to be both heavily attenuated and highly aggregated is
    unusual.}\label{figure:aneuploidy_comparison}}
{\includegraphics[width=0.6\textwidth]{c4_fig_comparison}} \end{figure}

Intriguingly, there are no obvious features that suggest why some proteins are
rapidly degraded whilst others are prone to excessive aggregation. Like
attenuated proteins, highly-aggregating proteins share features with NED
proteins and show indications of being core protein complex subunits, as
indicated by attributes such large interfaces and being highly coexpressed
within complexes (figure \ref{figure:aneuploidy_aggattrs}).

\begin{figure}[h] \includegraphics{c4_fig_aggattrs} \caption[Features of
    aggregating proteins]{\sffamily \textbf{Features of aggregating proteins} \\
    \small (A) Highly aggregation prone proteins form larger interfaces. (B) And
    are more highly coexpressed with other subunits within complexes. In both
    cases NA stands for not available, but this does not necessarily mean
    non-aggregating, since it could include proteins which only form aggregates
    in either wild type or aneuploid states. P-values calculated with Wilcoxon
    rank-sum tests.} \label{figure:aneuploidy_aggattrs} \end{figure}


\section{Discussion}
Supplementing the finding that NED predicts the response of proteins to
aneuploidy, these analyses go further in explaining the connection between
attenuation in aneuploid cells and non-exponential degradation in normal cells.
Both sets of results suggest that the assembly of a complex is not centred on
stable scaffold proteins, but rather on a core set of unstable, highly expressed
subunits. The first subunits to assemble tend to be rapidly degraded, with
significant stabilisation occurring upon binding, as indicated by the NED data
and again here. In contrast, the last subunits to bind tend to be comparatively
stable in both bound and unbound states. The take home point from this is that
both NED and attenuation behaviour are caused by rapid decay of excess protein
complex subunits - this is further supported by a recent study revealing highly
consistent observations with CNV proteins from tumour sample
\cite{Goncalves2017}.

The clearest difference between attenuation such as that seen in aneuploid cells
or tumours, and non-exponential degradation common to all cells, is in the fact
that NED is the result of relatively slight differences in protein expression
within complexes, whereas attenuation is the result of much larger increases in
expression affecting all of the proteins on affected chromosomes. However, there
are still important questions to be tackled in understanding the behaviour of
aneuploid cells - we have started to address one of these with a preliminary
study of changes in aggregation propensity caused by aneuploidy.

One important point to note is that classifying proteins as `aggregating' or
`highly aggregating' does not take into account proteins which simply don't
aggregate at all, or proteins which do not aggregate in wild type cells but do
in aneuploid. Similarly, whilst the aggregating fraction of proteins is
certainly much smaller, with the data we currently have it is not possible to
make quantitative comparisons of relative amounts of aggregated vs.
non-aggregated proteins.

Additionally, there are work-in-progress issues in distinguishing between
technical and biological noise in measuring the change in abundance of the
aggregating fraction of proteins. As the quantity of protein that can be
purified from the aggregating fraction of the cell is very small, there is
substantial noise in the measurements. The data from each experiment is
therefore normalised by mean-centring (see methods), such that the mean of data
points from each experiment is identical.

This reduces the variance in measurement of aggregate disomic ratios
(supplementary figure \ref{suppfigure:aneuploidy_aggnorm}), but comes at the
expense of being able to investigate the relationship between chromosome size
and aggregation propensity. For example, it is possible that duplicating larger
chromosomes leads to a greater overall increase in aggregation propensity across
the cell. Additional replicates of selected chromosomes are being planned to
better establish the extent to which noise in the data is attributable to
technical differences compared to biological ones.

The most pressing question that remains from this work is why some proteins are
attenuated whilst others aggregate. Work from other groups \cite{Pechmann2009,
Yang2012}, has shown that members of protein complexes are aggregation-prone.
Our data supports this, since we see that highly aggregating proteins are
enriched in heteromeric complexes and many attributes of aggregated proteins are
shared by NED and attenuated proteins. The simplest reason that explains why
some proteins are aggregated therefore is seems to be simply that they are less
efficiently degraded. Testing this hypothesis will be a priority as this study
goes forward.

\end{document}
