\documentclass[10pt]{report}

\begin{document}

\section{Introduction}
The Smc-kleisins are a highly conserved family of complexes present in all domains of life, with essential roles in the maintenance of chromosome architecture, cell division and DNA repair. Numerous regulatory proteins are recruited to these complexes, a subset of which are characterised by the presence of tandem HEAT repeat domains. In this work, through the use of a novel network-based approach, we demonstrate that these regulators share a close evolutionary ancestor that arose contemporaneously with eukaryotes. Furthermore, we overturn a long-held assumption about the presence of HEAT repeats in Nse5 and Nse6, two regulators of Smc5-6. In doing so, we show a clear split between cohesin and condensins one the one hand, which contain HEAT repeat proteins and are eukaryote-specific, and Smc5-6 on the other, which appears to be closer in several respects to prokaryotic Smc complexes. Our findings strongly support a common evolutionary origin for all the kleisin associated HEAT repeat proteins - including the NIPBL/Scc2 cohesin loader - which our analysis predicts to be interacting with kleisin. For this reason we propose that these regulators be collectively named Hawks, i.e. HEAT proteins associated with kleisins.

\end{document}
