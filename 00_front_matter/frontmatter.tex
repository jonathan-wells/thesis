\documentclass[a4paper,11pt,twoside,openright]{scrbook}

\usepackage{../jnwthesis}
\usepackage{amsmath}
\usepackage{lipsum}
\usepackage{standalone}
\usepackage{epigraph}  % For dedication and quote
\usepackage{bibentry}  % Allow full citations in text
\standalonetrue

\bibliography{/Users/jonwells/Documents/bibtex/Thesis}

\begin{document}

\chapter*{Declaration}
\addcontentsline{toc}{chapter}{Declaration}
\setlength{\epigraphrule}{0pt}
\setlength{\epigraphwidth}{\textwidth}
\epigraph{\normalsize This thesis presents my own work, and has not been
submitted for any other degree or professional qualification. Wherever results
were obtained in collaboration with others, I have clearly stated it in the
text. Any information derived from the published work of others has been cited
in the text, and a complete list of references can be found in the bibliography.
Published papers arising from the work described in this thesis can be found in
the appendices.}%
{\normalsize \vspace{0.8cm} -- Jonathan Wells, 2017}

% Dedication
\chapter*{}
\setlength{\epigraphwidth}{0.4\textwidth}
\epigraph{\textit{To my parents, Jane and Nick}}

\chapter*{Acknowledgements}
\addcontentsline{toc}{chapter}{Acknowledgements}
First and foremost, I would like to thank my PhD supervisor, Joe Marsh. Joe has
been a fantastic mentor, providing me with invaluable research opportunities,
sound advice and the freedom to pursue my own ideas, all the while with patience
and good humour. I must also thank my colleagues in the Marsh lab: Therese,
György and Marcin, for numerous useful and enjoyable discussions. Beyond
Edinburgh, I have had the good fortune of being able to collaborate with other
scientists from around the world, and whilst there are too many to name
individually, it goes without saying that this thesis would not have been
possible without them.

Even jobs as good as doing science for a living get tiresome sometimes, and in
these moments I am very grateful to have the friends I do; no problem is so
great that it can't be lessened by laughter and good company, both of which they
provide in abundance. Finally, I would like to thank my parents for all their
support over the last few years, and my brothers for being such great brothers.

% 1 page max
\chapter*{Abstract}
\addcontentsline{toc}{chapter}{Abstract}
Every living organism possesses a genome that contains within it a unique set of
genes, a substantial number of which encode proteins. Over the last 20 years, it
has become apparent that organismal complexity arises not from the specific
complement of genes per se, but rather from interactions between the gene
products - in particular, interactions between proteins. As an inevitable
consequence of the crowded cellular interior, most protein-protein interactions
are fleeting. However, many are significantly more long-lived and result in
stable protein complexes, in which the constituent subunits are obligately
dependent on their binding partners. Despite the abundance of protein complexes
and their critical importance to the cell, we currently have an incomplete
understanding of the mechanisms by which the cell ensures their correct
assembly.

In the chapters that follow, I have attempted to improve our understanding of
the regulatory systems underlying assembly of protein complexes, and the way in
which assembly as a whole affects the behaviour of the cell. The thesis opens
with an extended literature review covering the currently available methods for
characterising protein complexes. After this introduction, chapters
\ref*{chapter:operons}-\ref*{chapter:aneuploidy} are concerned with regulatory
mechanisms and biological implications common to the assembly of all protein
complexes. Chapter \ref{chapter:hawks} diverges from this work, and describes a
family of evolutionarily related proteins that regulate the behaviour of
condensins and cohesins.

Bacterial and archaeal genomes contain far less non-coding DNA than eukaryotes,
and coding genes are often packaged into discrete units known as operons. The
proteins encoded within operons are usually functionally related, either through
participation in metabolic pathways or as subunits of heteromeric protein
complexes. Since protein complexes assemble via ordered pathways, we reasoned
that there might be a signature of assembly order present in operons, the genes
of which are translated in sequential order. By comparing computationally
predicted assembly pathways with gene order in operons, we demonstrated this to
be the case for the large majority of operon-encoded complexes. Within operons,
gene order follows assembly order, and adjacent genes are substantially more
likely to share a physical interface than those further apart. This work
demonstrates that efficient assembly of complexes is of sufficient importance as
to have placed major constraints on the evolution of operon gene order.

Following this study of bacterial operons, I present results from research
investigating how patterns of protein degradation in eukaryotes are influenced
by the formation of protein complexes. This showed that, whilst most proteins
display exponential degradation kinetics, a sizeable minority deviate
considerably from this pattern, instead being more consistent with a two-step
degradation process. These proteins are predominantly members of heteromeric
complexes, and their two-step decay profiles can be explained using a model
under which bound and unbound subunits are degraded at different rates. Within
individual complexes, we find that non-exponentially decaying proteins tend to
form larger interfaces, assemble earlier, and show a higher degree of
coexpression, consistent with the idea that bound subunits are degraded at a
slower rate than unbound or peripheral subunits.

This model also explains the behaviour of proteins in aneuploid cells where one
or more chromosomes have been duplicated. In general, protein abundance scales
with gene copy number, so that the immediate effect of duplicating a chromosome
is to double the abundance of the proteins encoded on it. However, previous
analyses of mass spectrometry data, as well as my own, have shown that the
abundance of many proteins on duplicated chromosomes is significantly attenuated
compared to what one would expect. These proteins, like those with
non-exponential degradation patterns, are very often members of larger
complexes. Since the overall concentration of a protein complex is constrained
by that of its least abundant members, duplicating a single subunit will
predominantly increase the unbound, unstable fraction of that subunit. The
results from this work strongly suggest that the apparent attenuation of many
proteins observed in aneuploid cells is indeed a consequence of the failure of
these proteins to assemble into complexes.

Finally, I present a study concerning an important, universally conserved family
of protein complexes, namely the SMC-kleisins. Two members of this family,
condensin and cohesin, are responsible for two hallmarks of eukaryotic chromatin
organisation: the formation of condensed, linear chromosomes, and sister
chromatid cohesion during cell division. Unlike other SMC-kleisins, condensin
and cohesin possess a number of regulators containing HEAT repeats.  By
developing a computational pipeline for searching and clustering paralogous
repeat proteins, I was able to demonstrate that these regulators form a distinct
sub-family within the larger class of HEAT repeat proteins.  Furthermore, these
regulators arose very early in eukaryotic history, hinting at a possible role in
the origin of modern condensins and cohesins.

\chapter*{Lay Summary}
\addcontentsline{toc}{chapter}{Lay summary}
All cells are made up of a complex mixture of biological macromolecules,
including carbohydrates, lipids, and proteins. Proteins, the subject of this
thesis, are tiny, vibrating strings of amino acids with a strictly defined
sequence and three-dimensional structure. Every cell in your body, of which
there are some 50 trillion, contains further trillions of proteins that,
collectively, are responsible for carrying out virtually every biological
process you can imagine, from the moment you are born, to the moment you die.

However, although each protein is present in many copies in the cell, the full
set of unique protein species is comparatively small. What is more, the number
of protein-coding genes that an organism has bears almost no relationship to the
perceived complexity of that organism. For a humbling illustration of this,
consider the fact that your genome - that of a human - contains approximately
20,000 genes, whereas the pufferfish contains closer to 50,000 and even the
lowly banana has more than 36,000. What is it then about this collection of
genes that allows us to contemplate the difference between our selves and a
banana, whilst the banana just lies there, fruitily?

Part of the explanation stems from the fact that proteins interact extensively
with one another. Across the entire proteome (the collection of all proteins
present in a cell), a substantial fraction of proteins form stable complexes.
Haemoglobin, for example, is made up of two alpha and two beta globin subunits.
Without the tendency of proteins such as the globins to interact, the level of
complexity that we see across the spectrum of life - even in the simplest
microorganisms - would never be possible. However, we know very much less about
these protein complexes than we do about their constituent subunits. In
particular, we only have a basic understanding of how the cell regulates their
assembly, ensuring that proteins are produced at the right time and in the right
place.

In this work, I have attempted to explain why protein complexes matter, looking
at some of the ways in which the cell enables complexes to assemble, and what
the biological implications of this process are. For example, from studying the
organisation of bacterial genes, it is clear that the order in which their genes
are encoded closely matches the order in which the resulting protein complexes
assemble. This implies that these creatures must be under strong evolutionary
pressure to assemble protein complexes quickly and efficiently.

Concerning human biology, the later chapters describe a model of protein complex
behaviour that explains some of the features that we see in aneuploid cells -
that is, cells which have an abnormal number of chromosomes. This is a state
familiar to many of us as Down's syndrome, in which people affected have an
extra copy of chromosome 21. In closing, the work presented here, supported by
that of many others, demonstrates the fundamental importance of protein
complexes to life, and I hope goes some way to deepening our understanding of
their behaviour within cells.

\clearpage
\tableofcontents
\addcontentsline{toc}{chapter}{Contents}

\clearpage
\listoffigures
\addcontentsline{toc}{chapter}{List of Figures}

\clearpage
\listoftables
\addcontentsline{toc}{chapter}{List of Tables}
\clearpage

\chapter*{List of Acronyms}
\addcontentsline{toc}{chapter}{List of Acronyms}

\begin{acronym}[proteins]
    \acro{2D}{Two-dimensional}
    \acro{3D}{Three-dimensional}
    \acro{AIC}{Akaike information criterion}
    \acro{AMBER}{Assisted Model Building and Energy Refinement}
    \acro{AP}{Affinity-purification}
    \acro{CAPRI}{Critical Assessment of PRediction of Interactions}
    \acro{CASP}{Critical assessment of protein structure prediction}
    \acro{CHARMM}{Chemistry at HARvard Molecular Mechanics}
    \acro{CNV}{Copy number variant}
    \acro{DQE}{Detective quantum efficiency}
    \acro{ED}{Exponential degradation}
    \acro{EGTA}{Ethylene glycol tetraacetic acid}
    \acro{EM}{Electron microscopy}
    \acro{EPR}{Electron paramagnetic resonance}
    \acro{ESI}{Electrospray ionisation}
    \acro{eQTL}{Expression quantitative trait loci}
    \acro{GO}{Gene ontology}
    \acro{GTEx}{Genotype-tissue expression}
    \acro{HEAT}{Huntingtin, elongation factor 3 (EF3), protein phosphatase 2A
    (PP2A), target of rapamycin (TOR1)}
    \acro{iBAQ}{Intensity based absolute quantification}
    \acro{IR}{Isomorphous replacement}
    \acro{iTRAQ}{Isobaric tag for relative and absolute quantitation}
    \acro{LC}{Liquid chromatography}
    \acro{LECA}{Last eukaryotic common ancestor}
    \acro{LFQ}{Label-free quantification}
    \acro{MAD}{Multiple wavelength anomalous diffraction}
    \acro{MALDI}{Matrix-assisted laser desorption/ionisation}
    \acro{MAPS}{Monolithic active pixel sensors}
    \acro{MCL}{Markov cluster algorithm}
    \acro{MD}{Molecular dynamics}
    \acro{MS}{Mass spectrometry}
    \acro{NED}{Non-exponential degradation}
    \acro{NMR}{Nuclear magnetic resonance}
    \acro{PALM}{Photoactivated localisation microscopy}
    \acro{PDB}{Protein Data Bank}
    \acro{RSS}{Residual sum of squares}
    \acro{SASE}{Self-amplified spontaneous emission}
    \acro{SILAC}{Stable isotope labelling and culturing}
    \acro{SMC}{Structural maintenance of chromosomes}
    \acro{SMLM}{Single molecule localisation microscopy}
    \acro{STED}{Stimulated emission depletion}
    \acro{STORM}{Stochastic optical reconstruction microscopy}
    \acro{TAP}{Tandem affinity purification}
    \acro{TBM}{Template-based modelling}
    \acro{TEV}{Tobacco etch virus}
    \acro{TMV}{Tobacco mosaic virus}
    \acro{TROSY}{Transverse relaxation-optimised spectroscopy}
    \acro{Y2H}{Yeast-2-Hybrid}
    \acro{XL}{Cross-linking}
\end{acronym}

\chapter*{Notes on the use of published material}
\addcontentsline{toc}{chapter}{Notes on the use of published material}
This thesis is predominantly based on work that has already been published. I
present here a brief description of how each chapter relates to these published
articles, and the extent of my involvement in each case. In all cases, chapters
were rewritten and modified to varying degrees, using the published articles as
guides to structure where applicable. To make my contribution to each project
explicit, throughout this thesis I have used `I' to refer to analyses carried
out solely by myself, and `we' for those that were carried out in close
collaboration with others. Where figures have been adapted from published
material, notes indicating this have been made in the legends. Any papers for
which I am listed as an author that were released whilst working towards this
thesis can be found in appendix \ref{appendix:published}.
\\~\\
Chapter \ref{chapter:operons} is derived from:
\begin{quote}
    \fullcite{Wells2016}
\end{quote}
This paper has been rewritten, but is largely unchanged in terms of its content
and general layout, with the exception of some supplementary figures that have
been moved to the main body of text.
\\~\\
Chapter \ref{chapter:degradation} is based on and heavily adapted from:
\begin{quote}
    \fullcite{McShane2016}
\end{quote}
This was a large collaboration lead primarily by E. McShane and M. Selbach. A
detailed description of author contributions can be found in the paper; my own
contribution (with J. Marsh) was in the design, implementation and
interpretation of dry experiments investigating the relationship between protein
complex assembly and the observed patterns of protein degradation. It should be
noted that numerous analyses are included here which are not present in the
published version.
\\~\\
Finally, chapter \ref{chapter:hawks} is an extended version of the following
article:
\clearpage
\begin{quote}
    \fullcite{Wells2017}
\end{quote}
T. Gligoris initially suggested this project, experiments were designed by J.
Marsh and myself, and analyses were carried out by myself. Interpretation of the
data was predominantly my own, with contributions from all other authors.

\chapter*{}
\setlength{\epigraphwidth}{0.38\textwidth}
\epigraph{I do things like get in a taxi and say, ``The library, and step on
it.''}{\textit{-- David Foster Wallace}}

\end{document}
