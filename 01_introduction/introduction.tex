\documentclass[a4paper,11pt,twoside,openright]{scrbook}

\usepackage{../jnwthesis}
% \usepackage[T1]{fontenc}
\usepackage{amsmath}
\usepackage{lipsum}
\usepackage{standalone}
\standalonetrue

% Bibliography
% \usepackage[backend=biber,style=nature,backref=true]{biblatex}
\bibliography{/Users/jonwells/Documents/bibtex/Thesis}
% \let\cite\supercite

% Figures
% \usepackage{graphicx}
\graphicspath{ {../figs/wip/} }

\begin{document}

\chapter{Introduction}

\section{What are protein complexes?}
% In their deeply impressive book, The Origin and Nature of Life on Earth \cite{Smith2016}, Smith and Morowitz describe two of the crucial steps in the origin of life as being the emergence of a single autocatalytic network at the core of today's metabolic network, and the appearance thereafter of large oligomers. Amongst these oligomers are proteins, which are now an integral part of both cellular and viral life.

% NOTE: were the first proteins enzymes then? Again, Gyorgy's stuff on protein

Perhaps the most convincing and powerful description of the Earth's biosphere is the one given by Smith and Morowitz in their deeply impressive book, The Origin and Nature of Life on Earth \cite{Smith2016}. In this work, they argue that rather than trying to define life at the level of the organism, it should instead be defined in aggregate - at the level of the biosphere. To be alive therefore is simply to be participating in the planet-wide energy flux that channels geothermal and solar energy through biological matter. At the heart of this energy flux, and thus life, lies a core metabolic network shared by all extant organisms. In the emergence of life, they argue, a crucial moment was the emergence of this autocatalytic network - the first step in `lifting chemistry off the rocks'. This was followed by the emergence of the oligomer world, amongst which proteins are a first-class member; indeed, an incredibly ancient set of such proteins now catalyse the reactions that comprise the universal core metabolic network.

% NOTE: Point out that the core metabolic network and the protein universe are deeply intertwined - read up about evolution of cofactors. Gyorgy's ligand binding paper would be good here, especially if you can hint at the link between diversification of protein species and the metabolic network. Symmetry breaking or something like that, in that perhaps multichain binding sites in homomers could lead to heteromers. THIS IS (going to be) A GOOD PARAGRAPH, keep it but move till after you have defined homomers, heteromers etc. Why and how did protein complexes evolve, discuss Gyorgy's work, then link this into Lynch's non-adaptive arguments.

% NOTE: Alternate first paragraph
% All cellular life on Earth is built on a foundation consisting of four core groups of biological macromolecules, namely: carbohyrdates, lipids, nucleic acids, and proteins. At the lowest level of abstraction, these groups can be differentiated by their chemical formulas, but for the sake of simplicity it is generally more convenient to separate them on the basis of their cellular function. Broadly speaking, carbohydrates and lipids act as stores of chemical energy, with the latter having additional important roles in signalling and membrane formation. Nucleic acids, which comprise DNA and RNA, are principally involved in information storage and transfer. Proteins - the subject of this thesis - are involved in essentially everything else.

% TODO Improve and tighten up section about complexity. Check your maths! If not 21 then need to correct this for all subsequent calcs too (and in footnote).
Proteins, such as those just mentioned, are biological macromolecules comprised of tens to thousands of amino acids, each amino acid linked by covalent peptide bonds between their terminal amine and carboxyl groups. There are 20 amino acids that are commonly used by biological organisms, meaning that a 200 residue sequence can be composed in \(20^{200}\) different ways.\footnote{Or in base 10: \(1.6\times10^{260}\). By way of comparison there are approximately \(4\times10^{80}\) atoms in the observable universe. Most of space is vacuum, after all.} To all indications, this level of complexity does not seem to be limited by biological constraints, and the Shannon entropy of the currently known sequence space does not differ markedly from shuffled sequences, particularly if you include viral proteins. However, unlike nucleic acid, which functions primarily as information storage, proteins are not defined by their sequence so much as their structure in three-dimensional space. When we factor in bond angles between residues the total possible sequence-conformational space rapidly becomes incalculable. Fortunately, we can usually ignore much of this detail, instead thinking of proteins as simple biological units, described in terms of their form and function.

% NOTE *Define* homomers and heteromers - introduce them more explicitly.
Functionally, all proteins can be defined in terms of their interaction with other molecules in the cell. In the case of enzymes, these are typically small metabolites; however, since most proteins are not enzymes, most biologically interesting interactions are in fact with other proteins. The majority of these protein-protein interactions are fleeting, arising as a result of intracellular crowding, and may or may not be functional. Many though are more long-lived, and produce stable protein complexes, the formation of which is generally essential to the proper function of the constituent protein subunits. To discuss these complexes further however, it is first necessary to make some definitions. When discussing the quaternary structure of a protein, there are three overarching classes that encompass all of quaternary structure space. A single protein chain that mostly functions without forming stable interactions with other proteins is known as a \textsc{monomer}. More common however, is the case where interactions occur between identical copies of a single protein species - the resulting complex is known as a \textsc{homomer}. Finally, \textsc{heteromers} are formed from groups of distinct, non-identical proteins. All classes are common, though there are intriguing differences in their relative frequencies in prokaryotes vs. eukaryotes, an observation which I will return to later.
% TODO Remove that last sentence if you don't return to it later.

% TODO Paragraph on what proteins do, and why they do things that can't be done by monomeric proteins. Point out that Lynch's argument applies to homomeric proteins, but it is probably less applicable to heteromers
If stable protein complexes are so common, then what (if any) evolutionary fitness benefit do they confer on the cell? A frequently overlooked explanation is the non-adaptive one, such as that proposed by Michael Lynch in 2013 \cite{Lynch2013}. In this important paper he first notes the extensive literature showing how homomeric interfaces can be formed or destroyed by simple point mutations in the protein of interest, and the fact that proteins are frequently found in different multimeric states in different organisms. Armed with these observations, he describes a simple model in which transitions between multimeric states are represented as a Markov process, the probabilities of which are dependent on the underlying mutation rate and selection pressure. The implications arising from this are that under neutral to modest selection pressure (in either direction), mutations that promote homomerisation of a given protein will arise as a consequence of genetic drift.
% NOTE does this lead to a gamma distribution? can I show that this fits the observed distribution of proteins in the PDB?

% TODO New paragraph, what are the main advantages - allostery, particularly in homomers, reference Monod allostery paper (1st citation in Lynch2013). For heteromers, it perhaps brings together different functional units, e.g metabolic pathways. A facile example could be actin and myosin, which obviously need something to lever against to generate force.
The fact that non-adaptive explanations are often sufficient to explain seemingly complex biological phenomena is an important one, and provides a crucial null hypothesis that should be addressed before turning to adaptive (Darwinian) explanations \cite{Koonin2016}. Similarly, exaptation, the process by which new functions are generated from redundant or no longer necessary parts \cite{Gould1979}. Undoubtedly, however all of the above contribute in some degrees to the the ubiquity of protein complexes.

% NOTE: ugly last sentence, needs improving.
There are numerous potentially adaptive benefits provided by the formation of protein complexes. For example, when considering the metabolic cost of synthesising proteins, it may be more effective to split a large protein into parts, so that errors in translation are restricted to smaller units - modularity is a key feature of most robust systems, whether biological or designed. This is pertinent since the error rates in gene expression are sufficiently high as to present a major challenge to the viability of life in general. The literature on this topic is extensive, but for a good review focussing on error control in translation see Gingold and Pilpel, 2011 \cite{Gingold2011}. For more spectacular examples of situations in which formation of complexes has enabled novel evolutionary solutions, one only has to look at ATP synthase, which, being a fully functional rotary motor, could not possibly exist in its current state were it formed from a single polypeptide chain.

% NOTE Read the Monod paper and use their definition, not some crackpot one you've made up. Also, consider moving the allostery paragraph to a slighrtly later section, since its kinda interesting in that if allostery was such a clear adaptive benefit then you might expect precise allosteric mechanisms to be conserved - they are not always! Can you cite Joe and Tee's deer project here?
A particularly widespread phenomena arising from the formation of protein complexes is that of allostery. The original definition of allostery given by Monod, Changeux and Jacob \cite{Monod1963} referred to modulation of protein activity by small molecules binding away from the active site, but this definition has since been extended to include cooperative effects between proteins. The classic example of this is of course haemoglobin, in which the binding of oxygen to one subunit increases the binding affinity of neighbouring subunits by the propagation of structural changes through the subunit binding interfaces \cite{Perutz1976}. Intriguingly however, though one might expect beneficial allosteric mechansims to be highly conserved, it turns out that the opposite is often the case; even haemoglobin, the textbook example, differs in the exact mechanism of allostery in different species. This begs the question: is some degree of allostery inevitable, in the same way as genetic drift will in many cases lead unavoidably to the formation of homomers?

% TODO What are the aims of this thesis? Why does assembly matter? Needs improving, is an important paragraph. Maybe look for a more
Thus far we have discussed protein complexes as fully formed entities. However, as we shall see over the course of this work, the assembly process itself has great biological importance, and many phenomena can be better understood by taking it into account - the attenuation of protein levels observed in aneuploid cells (chapter 5) being a notable example. A central fact that underlies most of the work in this thesis is that the process by which protein complexes assemble is stochastic. However, whilst the binding of any two subunits is random - a product of Brownian motion and binding affinity - in aggregate, the assembly of protein complexes proceeds via non-random, TK pathways \cite{Levy2008a, Marsh2013, Macek2017}. This fact has profound implications for the behaviour of the cell.

% NOTE: IMPORTANT: Not sure quite how to lead into the review section, but possibly the historical overview would be nice. Important part is: "The following review is intended to give the reader a broad picture of the methods currently available for the study of protein complexes. The focus/emphasis? is not on methodological details, but rather on TK (design?) considerations and in how they enable different pictures of the protein world, not just at the narrow definition of individual protein complexes, but with a view to understanding how they fit into the larger proteome. Without these experimental advances, the work presented in the rest of the thesis would have been unthinkable." Then maybe a footnote about Donald rumsfeld's known unknowns.

\subsubsection{A brief overview of research on protein complexes}
% NOTE: does Molecular biology need capitalisation?
The tendency of proteins to form complexes and the functional implications of this have been understood since the early days of molecular biology. Though it is not clear who was the first to explicitly note the existence of protein complexes, it seems likely that the realisation arose in tandem with investigations into the nature of viruses. In 1935, W. M. Stanley reported the isolation of `a crystalline material which has the properties of tobacco-mosaic virus', and demonstrated that this was predominantly composed of protein \cite{Stanley1935}. It is not obvious that he understood this material to be a protein complex in the sense that we would think of it, but certainly by 1955 the Tobacco mosaic virus capsid had been clearly described as a self-assembling homomer comprised of several thousand identical subunits \cite{Fraenkel-Conrat1955}.

% NOTE: scrap the bit about phage group - it has nothing to do with perutz, and watson has nothing to do with protein complexes.
This period in time marks a turning point for the field of biology. In the years directly after the second world war many physicists became interested in biology. In 1945, Max Delbr{\"u}ck founded the Phage Group, from which a number of now famous researchers emerged. Amongst these was James Watson, who in 1953, along with Francis Crick, Maurice Wilkins and Rosalind Franklin, discovered the structure of DNA, thus ushering in the modern era of biology \footnote{ Rosalind Franklin was also involved in pioneering work using crytallographic electron microscopy to investigate the structure of protein-DNA complexes. Sadly, she died in 1958 at the age of 37, before achieving the recognition she deserved, but this later work eventually led to her proteg{\'e},  Aaron Klug, winning the 1982 Nobel prize for Chemistry. It seems likely therefore, that, had she lived, Franklin would have been the recipient of two Nobels. Needless to say, this would have put her in rareified territory indeed.}.
By the end of that decade, John Kendrew and Max Perutz had solved the structures of first Myoglobin \cite{Kendrew1958}, and shortly thereafter Haemoglobin \cite{Perutz1960} - the textbook protein complex.


% NOTE: When did mass spec appear - a return to non-structural methods?
Beyond structural methods, developments in mass spectrometry have led to a different, more nuanced understanding of the behaviour of proteins complexes and the wider proteome as a whole.

% NOTE: Plenty more to talk about with the advent of computational biology - work by Joe and colleagues, Lynch on non-adaptive hypotheses etc.

% NOTE: Conclude, and bridge to detailed discussion. It is on these non-structural methods that we will focus, but first an overview.

% NOTE: Scrap the historical section, and make sure you don't repeat yourself with the stuff about cryo-EM superseding X-ray crystallography.

\section{Structural characterisation of protein complexes}
Since the first post-war forays into the characterisation of protein complexes, technology has improved exponentially, and during this period the field of structural biology has been one of the most fruitful in all of science; X-ray crystallography in particular deserves special mention, having lead to no fewer than 14 Nobel Prizes since 1914. Of these Nobels, solving the structure of the ribosome - a huge complex consisting of dozens of protein and rRNA subunits - is perhaps the crowning achievement. However, whilst X-ray crystallography of most soluble monomeric proteins is now routine, there are still intractable difficulties for membrane proteins and large complexes. Recently, cryo-EM has seen a dramatic resurgence, driven by combined improvements in electron detectors and image processing approaches. Crucially, since it does not rely on being able to crystallise the sample of interest, cryo-EM is able to fill in many of the gaps left by hard-to-crystallise proteins. It is particularly well-suited to looking at large protein complexes and consequently is fast overtaking X-ray crystallography in this area.

\subsection{X-ray crystallography}
% NOTE: For many years, the structural biology was synonoymous with X-ray crystallography, and certainly X-ray crystallography is the method that first made the field a reality.
X-ray crystallography was the first method to make the field of structural biology a reality, and brought together three key technologies, each important in its own right. These technologies are: the ability to overexpress and purify proteins, the development of powerful sources of X-rays, and computational methods for resolving the diffraction patterns produced by firing X-rays at purified protein crystals. By and large, the ways in which X-ray crystallography can be used to determine protein structure are the same for monomeric proteins and those which form complexes. There are however some important differences and additional difficulties that are worth considering. Furthermore, although cryo-EM is poised to overtake X-ray crystallography as the method of choice for the solution of large heteromeric structures, there have been a number of exciting developments in the latter field that look set to ensure its future for many years to come. In the following section I will highlight of some of these advances, and attempt to give a summary of the current state of the field.

\subsubsection{Protein expression, purification, and crystallisation}
% NOTE: In this paper "Expression of protein complexes using multiple..." they talk about how there is no silver bullet for all protein complexes.
A generic protocol for the expression of a protein for cystallisation would involve transforming \textit{E. coli} with a plasmid containing your protein of interest, typically under the control of a strong, inducible promotor. Such a protocol works well for monomeric bacterial proteins, but expressing heteromeric protein complexes is often significantly more challenging. The key difficulty in the expression of heteromers lies in the production of sufficient quantities of pure sample, as in non-native host systems protein complex assembly is often inefficient or simply incomplete, making purification and subsequent crystallisation challenging. This problem - that of complexes being present in various oligomeric states - is one that also produces difficulties (albeit for different reasons) in cryo-EM projects.

At the simplest level, improvements in the cellular yield of bacterial heteromers can be achieved by considering the design of the expression vector, in light of the assembly pathway of the protein complex in question. As will be discussed in detail in chapter 2, the order of genes within protein complexes is under selection to match the assembly order of protein complexes \cite{Wells2016}. It has been demonstrated experimentally that taking this fact into account can markedly increase complex assembly efficiency, and that yields of heteromers in their fuly-assembled native state can be improved by using the native operon structure in expression vectors \cite{Shieh2015a, Poulsen2010}.

% NOTE: Need IMAC citation, review or otherwise. Also for DLS, and perhaps try and discuss it's use specifically for protein complexes?
When purifiying protein complexes there is a tradeoff between obtaining highly pure samples and ensuring that the intermolecular bonds between subunits are not disrupted. Though the diversity of methods for protein purification is bewilderingly high, in practice most methods suitable for protein complexes are variations on affinity purification. A standard, widely used protocol, mercifully summarised by Gr{\"a}slund et al. \cite{Graslund2008}, involves first generating a hexa-histidine tagged fusion protein. This `bait' protein is then expressed, preferably at native levels
\footnote{Somewhat counterintuitively, increasing the abundance of a single subunit may actually decrease the yield of the native complex. To understand this, imagine a trimer, assembled linearly as follows: A-B-C. If the concentration of subunit B were to be doubled, the resulting imbalance in stoichiometry would lead to A and C being preferentially sequestered in the form of A-B and B-C dimers, which are incompatible with the original trimeric structure. The idea that differentially modulating subunit expression within complexes can be deleterious is known as the balance hypothesis \cite{Papp2003}, and will be discussed extensively in coming chapters.},
and the resulting cell extract is put through a column containing immobilised metal ions, resulting in the capture of the 6-His tagged protein, as well as anything else bound to it. Upon washing the column, the purified protein complex will be retained on the beads, and can be cleaved off the metal-coated beads in subsequent washes. Ideally, this would be ready for crystallisation, but in practice multiple washes and additional purification steps are generally required before the sample is sufficiently pure. Unfortunately, these additional steps will affect the final yield, and also risk disrupting more delicate interactions with peripheral subunits. Methods such as Dynamic Light Scattering are now frequently used to assess sample purity in a non-invasive manner.

% CITATIONS PLEASE!
The above provides a good starting point for designing a purification protocol, but in many cases it will be necessary to tailor the process to the protein complex of interest. Depending on the orientation of subunits within the structure for example, different subunits may make better or worse bait proteins, as will N- or C-terminal histidine tags. Similarly, some complexes may be disrupted by the presence of the metal ions, in which case other beads, e.g. those coated in calmodulin, may be more suitable. Although there has been some progress towards high-throughput expression and purification pipelines, much of this work still relies on the expertise of individual structural biologists and research technicians.

Surprisingly however, the crystallisation process is still the main bottleneck in X-ray crystallography, despite having been largely automated by the development of screening robots. There have however been some important methodological developments in the crystallisation of membrane proteins, which will also be useful for many membrane complexes. In particular, an exciting new method - X-ray solvent contrast modulation - has enabled for the first time visualisation of the interaction between bulk membrane phospholipids and embedded proteins \cite{Norimatsu2017}. However, this method still requires good quality crystals, and these are obtained through trial and error - beyond a few general rules of thumb we still do not have a good understanding of how different proteins will behave under varying crystallisation conditions.

% NOTE: shorten - just say how the slightly underwhelming performance of the JPSI highlight the difficulty of expression and crystallography - this is still a major bottleneck. Actually, there is some dev of high-throughput systems. e.g "Tandem recombineering by SLIC cloning and Cre-LoxP fusion to generate multigene expression constructs for protein complex research."


\subsubsection{Diffraction pattern acquisition}
% NOTE: bragg could do with a citation probably.
Once suitable crystals have been obtained, the main hurdle has been hurdled and image acquisition can begin. In contrast to earlier steps, enormous progress has been made in this domain since William L. Bragg first demonstrated that crystals diffract X-rays in 1913. By far the most important development in this domain has been that of synchotron X-ray sources. Synchotrons are able to produce X-rays at far higher intensities than traditional sources, and as such greatly reduce the time it takes to produce diffraction patterns. Additional properties of the X-rays can also be manipulated, for example narrowing the beamline in order to focus on the best quality region of the crystal, thus improving the resolution that can be achieved from the diffraction pattern.

% NOTE: Peak "beam" energies sounds kinda silly since its a pulse, not a beam. Depending on space, it might be nice to mention how you don't have to worry about cryocooling either - Ada Yonath was one of the first to see the benefit of this in solving the structure of the ribosome. Also, it might be possible to use this for non-crystalline imaging? Is it OK to use a colon in "this becomes a non-issue:?"
More recently, X-ray free electron lasers (XFELs, fig. \ref{c1fig1}) have also begun to make an appearance in structural biology. It is hard to overstate the impact that this technology will have on the field, since XFELs are capable of producing peak beam energies approximately ten orders of magnitude greater than 3rd generation synchotrons \cite{Shi2014}, and in doing so enable a radically different approach to crystallography. The principle benefit of this additional power is that the time needed to generate a diffraction pattern is drastically reduced: from hours to femtoseconds. A crystal in the path of such high-energy photons will be vaporised, but since the diffraction pattern will be obtained faster than the sample is destroyed, this is a non-issue: a fact first noted by Neutze et al.\cite{Neutze2000}, giving rise to the term `diffraction before destruction'. However, this generates a need for a great many crystals, but in practice this too turns out not to be a problem either, since these crystals need only be a few nanometres in size. In fact, since nanoscale crystals are far easier to grow, the method also circumvents the tedious trial and error process of producing mesoscale crystals.

\begin{figure}[h]
    \includegraphics{c1_fig1_xfel_v3}
    \caption[X-ray free electron lasers]{\sffamily \textbf{X-ray free electron lasers} \\ \rmfamily These here are electron lasers, they be very nice}
    \label{c1fig1}
\end{figure}

\subsubsection{Structure determination}
% NOTE: The structure of a protein in a crystal can be determined from its electron density function. This is a function of both the phases and amplitude of the x-ray diffraction maxima.

Interpretation of the crystal diffraction pattern required the solution of a long-standing obstacle in the early days of X-ray crystallography, known as the phase problem. The phase problem exists due to the fact that, whilst diffraction patterns capture the amplitude of diffracted photons from a crystal (seen as the intensity of spots on the photograph), the phase of those photons is lost in the process of image acquistion. Unfortunately, it is the phases of the diffracted photons, rather than their amplitudes, that carry the most information about the underlying crystal structure. The eventual solution of this problem by Max Perutz was the key to his and Kendrew's determination of the first protein structures.

Perutz's breakthrough came when he realised that a technique previously used for phasing much crystals of much smaller molecules could also work for proteins. This method, known as isomorphous replacement (IR)  \cite{Robertson1936}, incorporates a heavy metal into the crystal, but (crucially) does not significantly alter the structure of the underlying protein. As a result, the position of spots in the diffraction pattern remain almost unchanged, but subtle differences in their intensity point to the location of the heavy metals, thus providing a reference point for calculation of the X-ray phases. This method has since been followed by several others, most notably Multiple wavelength anomalous diffraction (MAD). This method operates on different principles from IR but is popular since it is limited only by the quality of the diffraction pattern provided to it. For large protein complexes, polynuclear metal clusters are often used in place of individual heavy atoms because of their particularly electron density and associated isomorphous or anomolous scattering signal \cite{Dauter2005}. This approach has recently been used to good effect in solving the structure of the notoriously difficult mediator complex \cite{Nozawa2017}.

% NOTE: Have you cited PDB yet? And will you discuss homology modelling?
As a consequence of the ever-expanding number of structures in the Protein Data Bank and the widespread availability of sequence data, it is usually possible nowadays to avoid \textit{de novo} phasing altogether. Molecular replacement makes use of the fact that closely related sequences generally have very similar folds, and therefore can be used as a template to guide brute-force solution of the phase problem. There are currently many programs that automate this process - for example, Phaser \cite{McCoy2007}, which is available within the widely used CCP4 software suite \cite{Winn2011a}.

% NOTE: Need to find the right place to discuss the difficulty of distinguishing crystal contacts from biological ones.

\subsection{Single-particle cryo-electron microscopy}

X-ray crystallography has been, and will continue to be, an enormously useful too for investigating proteins and protein complexes. However, in recent years, a resurgence in an old technique has had an major impact on structural biology, and in particular on our ability to solve the structures of large protein complexes above 300 kDa in size (approximately 3000 residues). It's TK (unique suittability?) for larger structures is particularly useful since these often prove prohibitively difficult to crystallise, in large part due to compositional heterogeneity of the purified samples, which cryo-EM can more easily handle. The two methods are therefore highly complementary, and indeed many structures are solved to high resolution by a combination of the two - cryo-EM for the coarse grained strucutre, and X-ray crystallography for atomic resolution of individual subunits.

However, as interest in cryo-EM increases (in March 2017 the Wellcome Trust announced a £20M grant for cryo-EM equipment in several UK laboratories), there are signs that the field is gaining ground on X-ray crystallography. In June 2016 two important barriers were broken, with a paper in Cell simultaneously describing the structures of two homomeric complexes: isocitrate dehydrogenase and glutamate dehydrogenase. The former weighs in at just 93 kDa, and is the first single-particle cryo-EM structure of a <100 kDa complex, while the latter was resolved to 1.8Å, breaking the <2Å barrier \cite{Merk2016}. As we shall see, the remarkable technological achievements displayed in this paper and several others from the last few years have been been driven by dramatic improvements in two areas\cite{Bai2015}. Even with this rapid progress however, there is good reason to believe that further reductions in the limits of resolution are possible.

% NOTE: might need to rethink this section title in order to incorporate phase plate stuff?
\subsubsection{Direct electron detectors}
A major development in cryo-EM came with the replacement of photographic film by digital direct electron detectors, specifically monolithic active pixel sensors (MAPS). Surprisingly, it was not until relatively recently that digital detectors came into widespread use, as for a long time they had unfavourably low detective quantum efficiencies (DQE) compared to that of film \cite{McMullan2009}. DQE is a measure of the signal to noise ratio that can be achieved relative to an ideal detector \cite{Dainty1975}, and is defined as follows:
\begin{displaymath}
    DQE = (S/N_{in})^{2}/(S/N_{out})^{2}
\end{displaymath}
Where \begin{math} S/N_{in} \end{math} and \begin{math} S/N_{out} \end{math} are the input and output signal-to-noise ratios respectively. A DQE of 1 would therefore imply that the detector was not responsible for any noise in the final image. For reference, film has a DQE of around 0.3, whereas the current state-of-the-art MAPS detectors achieve roughly twice that.

Now that the DQE of MAPS detectors has surpassed that of film, several other compelling advantages can exploited. From a practical standpoint, they are significantly faster to use, since images can be viewed immediately after collection and their acquistion can be automated. More importantly however, they can be operated in counting mode, where instead of integrating the signal produced by each incident electron across all the pixels in which a charge was registered, only the pixel with the highest charge is counted \cite{McMullan2009a}. This is similar in principle to the way in which optical microscopy techniques like PALM \cite{Betzig2006} and STORM \cite{Rust2006} achieve super-resolution images, and the company Gatan has recently developed the idea further with the introduction of a super-resolution mode for their K2 Summit detector.

% NOTE: I think there's still some pretty substantial innacuracy here - go through it and tighten up. Also, GPCR structure released just this year in nature.
One new technology which is begining to make an appearance is the phase-plate, which can be used to produce phase contrast during image acquisition. In order to be able to correctly distinguish different particles in the sample it is important to have good contrast in the images. Unfortunately, the method by which this contrast is currently changed relies on defocusing the image slightly: as a result, if greater the contrast required, it comes at the expense of eventual structure resolution. The Volta phase-plate circumvents this issue by modulating the phase directly, without affecting the focus of the image \cite{Danev2014}. Though the principle has been around for some time, it was not until recently that various practical issues were solved, enabling them to produce a 3Å structure of the 20S Proteasome, thus matching the resolution achieved by the defocus method \cite{Danev2016}.

\subsubsection{Image processing and structure determination}
A second important factor in cryo-EM's recent success has been the appearance of better image processing software, which has enabled researchers to get the most out of the concurrent improvements taking place on the hardware side. In addition to improving resolution, the emergence of electron detectors capable of producing high frame-rate videos in counting mode also had a secondary benefit, in that it enabled beam-induced motion blurring in the images to be corrected computationally, a feat that was first achieved by two groups almost simultaneuously in 2013 \cite{Bai2013, Li2013}. Since the reduction in signal quality incurred by beam-induced movement is around five-fold if uncorrected \cite{Henderson1985}, this was a highly significant breakthrough, and is now a standard procedure that can be performed using the widely used RELION software \cite{Scheres2012,Scheres2014}.


% NOTE: reduce the dimensionality of the classification prob - is this correct? Still needs improving and references. Citations available from main scheres cryo-em review.
A second area in which computational improvements have occurred is in image classification (fig. \ref{c1fig2}). In 3D single-particle cryo-EM, individual protein complexes are fixed in random positions and orientations in the flash-frozen sample - to determine the structure, each particle captured in the imaging process must first be categorised according to its orientation. However, this is made difficult by compositional heterogeneity in imperfectly purified protein complexes, and also by asymmetry in structures. For symmetrical structures, such as the two highlighted at the begining of this section, the number of particles required in the image is usually considerably lower, since multiple axes of symmetry reduce the dimensionality of the different orientations that must be distinguished between. Interestingly, when dealing with samples that are 2D crystals (electron crystallography, rather than microscopy), it is still necessary to get images of particles in different orientations, but these are acquired tilting the sample itself.

\begin{figure}[h]
    \includegraphics{c1_fig2_cryoem}
    \caption[Image classification in cryo-EM]{\sffamily \textbf{Image classification in cryo-EM.} \\ \rmfamily 2 Panels, one showing how individual proteins can be grouped into different orientations, a second panel showing how compositional heterogeneity can cause issues?, do something with pretty pictures and then put a gaussian noise filter over the top or something.}
    \label{c1fig2}
\end{figure}


% TODO
Dealing with compositional heterogeneity is a separate, harder problem, but has been effectively dealt with, first by max likelihood methods, and more recently within the bayesian framework of the RELION program mentioned previously. It is worth noting that this open-source software was initially written almost entirely by Sjors Scheres, who has played an essential role in bringing the cryo-EM field to where it is today.

% NOTE: Important papers to read through:
% Wiered2016 - DNA-protein complexes
% Morris2014 - NatProtocols, AP-MS
% NOTE: Also, what's a totally classical way of doing it? Co-IP with western blotting and SDS-PAGE.
\clearpage
\section{Non-structural characterisation of protein complexes}
Thus far, I have only discussed some of the structural methods that can be used to describe protein complexes. However, there is a great deal of useful information that cannot be determined solely from the molecular structure of a protein complex; for example, the cellular abundance of that complex and its constituent subunits. Attempts at non-structural characterisation of protein-protein interactions began in 1989 with the development of the yeast-2-hybrid assay (Y2H) \cite{Fields1989}, in which two proteins of interest are fused to a DNA binding domain and a transcriptional activator domain, allowing binary interactions (or lack thereof) between the proteins to be detected by the expression of a reporter gene. This assay has been very successful, with the original paper having been cited over 6850 times since publication; most recently, it has been used in a high-throughput manner to map the binary interaction landscape of \textit{E. coli} \cite{Rajagopala2014}, producing a map of 2,334 pairwise interactions and enabling inference of many novel protein complexes in the process.

% NOTE: Need to make it clear that mass spectrometry is useful for far more than just studies of the interactome, and indeed mass-spec will probably become to amino-acids what sequencing is to nucleic acid.
However, though simple and cost-effective, there are inherent limitations to the technique: most obviously, the use of bulky reporter domains for example risks disrupting or preventing subtle interactions between many proteins. As a result, approaches using mass spectrometry have largely superseded Y2H as the method of choice for quantitative studies of the interactome. Through technological innovation and clever experimental design, mass-spectrometry has proven to be highly versatile, and has been used for a number of different purposes, including elucidation of protein complex assembly pathways \cite{Levy2008a,Marsh2013}, investigations into the evolutionary history of complexes \cite{Wan2015}, and generation of richly detailed interactome datasets \cite{Hein2015}.

\subsection{Cross-linking mass spectrometry}
The first application of mass spectrometry that I will discuss is one that can be thought of as a hybrid between structural and non-structural technologies. Cross-linking mass spectrometry (XL-MS) provides information on the constituent parts of protein complexes, but can also be used to produce low-resolution structural information in the form of distance constraints between residues from different subunits. For this reason, it is particularly effective when used in combination with more established structural techniques and as such has become an important part of an emerging, integrative approach to structural biology \cite{Stengel2012,Ward2013}.

The best example of this approach to date is one in which XL-MS plays an important role, namely the ongoing effort to understand the structure of the nuclear pore complex (see Beck and Hurt, 2016 \cite{Beck2016}). Due to their formidable size (\textasciitilde 120 MDa in humans, compared to \textasciitilde 3.5 MDa for the ribsome) and high degree of compositional variation between species, it is often not possible to distinguish between specific subunits, many of which are paralogues; in such cases, XL-MS can, for example, provide important information about their proximity, and in doing so enable identification of ambiguous subunits within the cryo-EM electron density map \cite{Bui2013}.

\subsubsection{Chemical cross-linkers}
Much of the power of XL-MS comes from the availability of a wide variety of different cross-linkers that impose specific distance constraints on the interactions that can be probed. Similarly, the biochemical specificity of these linkers can be used to look at interactions between specific functional groups. Most commonly used are homobifunctional cross-linkers that join primary amines \cite{Leitner2016}, i.e. lysine residues or N-termini, with spacer arm lengths ranging from \textasciitilde 3Å to \textasciitilde 35Å. Plenty of alternatives exist however, allowing more nuanced approaches to the problem of interest. Heterobifunctional linkers (in constrast to homobifunctional ones) allow different groups to be targeted, for example joining amine to carboxyl (aspartate, glutamate, C-termini) groups.

% NOTE: Need to read this 2016 review by Wiered and Mann on chromatin/TF complexes. They don't specifically mention XL-MS but important nonetheless. Even if you don't end up citing it specifically or mining it for a few references, this wopuld be another thing to discuss as something that I haven't been able to cover - the protein-nucleic acid complexes. Also in this review, it mentions the importance of accurate quantification for distinguishing between specific and non-specific binders. Think this is more closely related to AP-MS though.
More exotic linkers allow for...


By virtue of the fact that XL-MS provides information at the level of specific protein interfaces, it is especicially well suited to investigating individual protein complexes and interactions. However, the power of mass-spectrometry as a whole is its high dynamic range: it can be used to probe intimate mechanistic details of specific protein complexes, but also to look at entire proteomes in on a practical timescale. This is something that is beyond most structural techniques. We will next look at affinity purification mass-spectrometry (AP-MS), which can be used to great effect at this larger scale.

% NOTE: See if you can include a warning about the pitfalls of mass spec, and why caution is particularly important. To do this, cite the Journal of proteome research paper criticising the draft proteomes by way of olfactory receptors. Also theres another one kicking about somewhere I think?

% NOTE: Good God, there's also hydrogen-deuterium exchange mass spec, which lets you see solvent accessible and inaccessible regions. You're killing me here.

% NOTE: AP-MS is starting to be combined with XL-MS apparently - would make it possible to see interaction surfaces on a proteome scale.
\subsection{Affinity-purification mass spectrometry}
In its simplest guise, AP-MS enables the identification and quantification of the interaction partners of a given protein. The general principle is as follows: a column containing beads capable of capturing your bait protein is prepared. Native cell extract (though sometimes over-expression of the protein of interest is required) is then washed over the column, leading to the capture both the bait and proteins bound to it via co-immunoprecipitation. The eluent is generally subjected to peptide fractionation, and mass-spectrometry is then used to quantify either the relative or absolute abundances of members of the purified complex. For high-throughput studies, multiple proteins are used as baits, enabling large interaction maps to be generated.

% NOTE: the paragraphs that follow are limited to those variations of the technique most pertintent to the study of protein complexes.
Though conceptually simple, AP-MS is an enormously powerful technique, and one that deserves more attention that it is going to get here. Fortunately, several reviews have been written on the topic, and I therefore direct the reader to these \cite{Oeffinger2012,Morris2014,Aebersold2016}; in particular, that by Morris and colleagues is excellent \cite{Morris2014}. For the sake of brevity, the paragraphs that follow are limited to just the most important variations of a method that has been instrumental in achieving our current understanding of the protein interactome \cite{Malovannaya2011,Hein2015,Huttlin2015,Wan2015}.

% NOTE: Perhaps briefly discuss the hein paper as a case study -
\subsubsection{Single-step versus tandem affinity purification}
Historically, there have been two main approaches to affinity purification - single-step and tandem affinity purification \cite{Rigaut1999} (TAP) (fig. \ref{c1fig3}). In the former, the bait protein is either expressed under completely endogenous conditions and captured using antibodies, or expressed with a tag such as green fluorescent protein \cite{Hubner2010} and captured using methods appropriate to the tagging system. In contrast, TAP makes use of a specific TAP-tag, which consists of a protein A domain and a calmodulin binding peptide, linked by a tobacco etch virus protease cleavage site. This tag enables a two-step purification procedure, first by capture of protein A on immunoglobulin-G beads, followed by protease cleavage and recapture on calmodulin beads. This second washing step allows for very stringent purification of complexes.

\begin{figure}[h]
    % \includegraphics{c1_fig1_xfel}
    \caption[Common approaches to affinity-purification mass spectrometry]{\sffamily \textbf{Common approaches to affinity-purification mass spectrometry.} \\ \rmfamily 3 panels showing AP-MS, single step with endogenous and tagged, tandem affinity, and if there's space or you can think of a nice way of representing it, affinity enrichment.}
    \label{c1fig3}
\end{figure}

TAP neccesarily requires tagging of the bait protein, but in the single-step procedure it is possible to avoid this if desired. The are some straightforward trade-offs to consider when deciding whether to use endogenous or tagged proteins: For non-tagged, endogenously expressed baits, you have the benefit of capturing the protein in its native state. However, this comes at the substantial cost (both in time and money) of having to raise specific antibodies against the protein. Furthermore, there are difficult issues associated with cross-reactivity and specificity when using antibodies, particularly in studies where multiple proteins are being targeted. Though there are methods that attempt to deal with these issues (most notably QUICK \cite{Selbach2006}), in the majority of large-scale studies tagging of bait proteins is likely to be more practical.

% NOTE: Can you cite some papers spanning the range of time from first TAP studies to presentish? Also, don't like that last sentence - it's too presumptuous.
Whether using endogenous or tagged proteins, there are some fairly compelling advantges to using single-step procedures in general over TAP. TAP has been an important technique for almost 20 years, but its raison d'être was to remove as many possible contaminants or non-specific interactors from the purified protein complex. This was neccessary in the early days of mass spectrometry, since it was not possible to quantify protein abundances, and thus contaminants in the sample would be erroneously annotated as members of the protein complex. However, there have been enormous improvements in the sensitivity of mass spectrometers since TAP was first described. With these improvements, particularly in the area of label-free quantification, there has been a growing appreciation of the importance of weak, non-obligate interactions between proteins \cite{Perkins2010a,Hein2015}. By design, TAP removes these weak interactors, and thus its utility is becoming increasingly restricted, at least for studies of the interactome. Therefore, unless equipment is an issue, single-step procedures combined with accurate quantification should be considered preferable to TAP where possible, particularly for large-scale studies.

\subsubsection{Quantification of protein abundances}
The emergence of quantitative mass spectrometry, via both label based and label free-methods, has had a transformative effect on the field of proteomics. For our purposes, the principal benefit arising from the ability to quantify protein abundances is that it allows the stoichiometry of protein complexes to be determined. This is essential for distinguishing obligate interactions from transient ones, and - more generally - for providing a complete characterisation of the complex. The difficulty in using mass spectrometry as a quantitative tool is that, whilst the location of peaks on the spectrum allows identification of peptides, peak intensity alone is not sufficient to determine abundance. Label-based methods such as SILAC \cite{Ong2002a} and iTRAQ \cite{Ross2004} (amongst others \cite{Gygi1999,Thompson2003}) allow for either relative or absolute quantification (through metabolic incorporation of amino-acid isoptopes in the case of SILAC, and N-terminal isobaric tags in iTRAQ).

A significant drawback to label-based methods is their cost, which can be prohibitive. An alternative approach is label-free quantification (LFQ), which in general rely on either spectral counting \cite{Liu2004,Zybailov2005} or peak intensity-based algorithms. Spectral counting is a conceptually simple, semi-quantitative approach, and has been widely used (possibly abused \cite{Lundgren2010} - `semi' being the operative word). Intensity-based algorithms undoubtedly offer more accurate quantification; for the interested reader, comparative analyses and reviews of several available methods are available \cite{Nahnsen2013,Fabre2014}. One recently developed algorithm of note that has been enthusiastically received by the community is MaxLFQ \cite{Cox2014}, which is available as part of the larger MaxQuant software package \cite{Cox2008}.

% NOTE: They definitely (?) use this in the Hein paper, but what about the Wan one? Seems they came out at around the same time, so perhaps they arrived at a similar method more or less independently.

% NOTE: How to put it all together? AE-MS is way too specific - has only been cited a few times, though it is very new. Perhaps talk about fractionation instead. Fractionation vs pre-fractionation?
\subsubsection{Affinity enrichment mass spectrometry}
A recent development in the field of AP-MS - one with particular importance for the study of protein complexes - is that of affinity enrichment mass spec (AE-MS) \cite{Keilhauer2015}. Rather than being a technological break-through in any one area, this instead combines several of the elements we have been looking at in this section.

% NOTE: Discuss the importance of quantification, particularly relative vs absolute, SILAC vs iBAQ. This is the place to discuss the pitfalls of mass spectrometry. Need to at least skim those spectral counting papers. Also go through the affinity enrichment paper intro with a fine-tooth comb for some useful references for the next section. https://www.coursera.org/learn/experimental-methods/lecture/n7Z1J/lecture-3-quantification-in-proteomics

% NOTE: Bleurgh
\subsection{Inferring protein complexes from interaction networks}
spoke and matrix models, socio-affinity scoring. Read panorama paper. \cite{Wan2015}.

\clearpage
\section{Computational prediction of protein complex structure}
% NOTE: Perhaps start with a very short paragraph talking about how computational analysis is integral to structure characterisation, both for structural methods and non-structuralm, inferring complexes from interaction data (spoke affinity etc etc models.) Then lead into the prediction from sequence.
Frequently, experimental methods for determing protein complex structure are not possible, or tellingly, are simply no longer the best use of a researcher's time and money. Prediction of protein structure from sequence, first prophesised by Anfinsen in 1973\cite{Anfinsen1973}, has been a long-standing challenge in biology which has until recently been impossible in practice, for want of both sequence data and computational power. However, the genomic era has seen exponential increases in both of these areas, along with a similar expansion in the number of experimentally determined protein structures. Concomitantly, there has been a significant improvement in our ability to predict structures computationally.

Broadly speaking, the field of protein structure modelling can be separated into two overlapping subgroups, separated more by philosophical standpoints than genuine differences. Older, and currently more practical, are top-down approaches based on the use of templates for the structures being modelled,  which are typically selected based on close sequence similarity with the target protein or complex. In contrast, there is equally great interest in prediction of structure from first principles - an approach that is exemplified by the field of molecular dynamics. Template-based modelling (TBM) and molecular dynamics represent two opposite sides of the protein modelling community, but in practice there is a great degree of overlap between the two, and most of the methods the methods that I will describe below borrow elements from both, as is the case in molecular docking, which aims to model protein-protein interfaces.
%
% Although enormously useful, homology modelling is limited by the requirement for template structures that are a close sequence match to the protein of interest. This limitation is apparent in the case of membrane proteins and other difficult-to-solve structures, which due to their relative scarcity in the PDB are also less amenable to homology modelling \cite{Reddy2006}. In addition, without the aid of experimental data or additional computational tools, homology modelling alone is less able to predict protein-protein interactions, and thus protein complexes in general. For this reason, there is still great interest in ab initio, template-free prediction of structures, as well as methods for predicting protein-protein interactions \cite{Bonvin2013,Moult2016}.

\subsection{Template-based methods for structure prediction}
The extent of the improvement in predictive power is such that, for sequences with close sequence similarity to known structures, it is usually possible to produce structures that are within a few ångströms of the experimentally determined version, as measured by root-mean-square-deviation of residue distances and other metrics \cite{Haas2013,Moult2016}. This process is known as template-based modelling (TBM), and nowadays is routinely used to facilitate experimental structure determination of single protein chains.

% NOTE: Can you find any better references in the Szilagyi review?
For complete protein complexes, regardless of whether the structures of individual subunits are already known, it is often possible to reach a realistic approximation of the correct protein complex structure by a combination of homology modelling and molecular docking. Though most applicable for investigating protein-ligand interactions, molecular docking combined with homology modelling is becoming increasingly viable for protein-protein interactions, as evidenced by numerous recent studies and the results from the CASP and CAPRI competitions \cite{Jiang2013,Rajapaksha2014,Agostino2016,Lensink2016}.

\subsubsection{Template-based modelling}
TBM is based on the principle that the degree of sequence divergence in homologous proteins is closely related to their structural similarity \cite{Chothia1986}. Once a suitable template is found - typically with at least 40\% sequence identity with the target protein - the sequences are aligned and conserved regions are used to map fragments of the target onto the template structure. This is followed by replacement of the loop regions (which tend to be less well conserved) and additional refinement procedures.

There are numerous methods based on extensions of this basic protocol that enable modelling of complete protein complexes, in addition to individual subunits \cite{Chen2008,Tuncbag2011,Guerler2013}. One important and widely used strategy for template identification is threading, or dimeric threading, in the case of modelling complexes \cite{Bowie1991,Lu2002}. Threading differs slightly from approaches based solely on sequence homology, in that it relies more on fold recognition than sequence similarity - this is assessed by a scoring function - the template that is eventually selected is the one which minimises this function. As a general rule, threading is used when the target sequence has particularly low sequence similarity with other known proteins, but in practice, most modern software takes these decisions out of the hands of the user. For a much more comprehensive overview of the software and underlying strategies behind TBM, readers should see the recent review on the topic by Szilagyi and Zhang \cite{Szilagyi2014}.

% NOTE: de novo docking and template driven docking.
\subsubsection{Prediction of protein-protein interfaces}
There are several software tools available that are designed to predict intermolecular residue contacts that are formed between two proteins (often referred to as `docking'), with RosettaDock \cite{Lyskov2008} and HADDOCK \cite{Dominguez2003,VanZundert2016} being amongst the most popular.

\subsection{De novo structure prediction}
The docking methods described above are contingent on having structures available for the proteins whose interactions you are trying to model. However, is often the case that there is no experimentally solved structure or suitable template for homology modelling. In the past, this would have meant that the best one could do would be to try and predict secondary structure regions and infer the presence of binding domains from homologous sequences using tools such as JPred4 \cite{Drozdetskiy2015} (for secondary structure) or databases such as PFAM and Uniprot \cite{Finn2016,Consortium2017} for domain predictions and functional annotations. However, there are promising signs that prediction of protein structure from first principles alone will one day be possible, and in other areas important work is being done that circumvents the substantial difficulties entailed by truly ab initio methods.

\subsubsection{Using protein coevolution to infer intermolecular contacts}
Using the evolutionary sequence record to inform structure prediction is an idea that has been around for at least 20 years \cite{Altschuh1987}, but has been prevented from becoming practical by the fact that it is very challenging to distinguish coevolving sites that indicate true amino acid contacts from those that are transitive. For example: if residues A and B are in contact, and residues B and C are in contact, then A and C may also show a strong coevolutionary signal, despite interacting via an intermediary residue. Over the entire protein sequence, this blurring of coevolutionary signal is sufficient to prevent meaningful structure prediction. The major breakthrough in tackling this problem was achieved with the developement of an algorithm named Direct Coupling Analysis \cite{Weigt2009,Lunt2010}, which extends Shannon's concept of mutual information \cite{Shannon1948} and enables direct and indirect residue contacts to be distinguished from eachother. This was then implemented in a more generally applicable and user-friendly format primarily by Deborah Marks, enabling the method's widespead use in the structure-prediction community \cite{Marks2011,Marks2012,Hopf2014}.

% NOTE: Need a citation or two for the stuff about heteromeric interactions arising from gene duplications.
Though originally used for single protein structures, is equally applicable to protein complexes, as intermolecular contacts are subject to many of the same coevolutionary pressures as intramolecular ones. Thus, the EVcouplings method has recently been successfully applied to protein complexes \cite{Hopf2014}. Out of a set of 82 protein complexes with unsolved structures, 32 had a sufficiently good sequence record as to be able to predict the entire complex de novo, whereas others were sufficient to predict intermolecular contacts, but not the entire structure. Unfortunately, a major limitation of this technique is identification of homomeric contacts, since without additional information they cannot be distinguished  from intramolecular interactions. A releated issue is that many nominally heteromeric interactions arise from homomeric interactions between genes which have undergone duplication and and subsequent genetic drift \cite{Wagner2001,Wagner2003,Fokkens2012}. In such cases, it may not be possible to acquire a sufficient number of sequences for structure/interaction prediction, particularly if the proteins in question have diverged recently.

\subsubsection{Molecular Dynamics}
Saturation of structure and sequence space is a long way off, as is made clear from recent studies sampling viral and prokaryotic genomes \cite{Brum2016,Shi2016,Mukherjee2017}. As such, a substantial proportion of the protein universe is beyond the reach of either homology modelling or methods using information from the evolutionary sequence record. The field of molecular dynamics, which is

% NOTE: Noxclass - distinguishing obligate and transient interactions. Assembly order predictions, lifespan and turnover etc. Integrating into the wider interactome. flexibility etc RadivojacP,Obradovic Z,SmithDK,ZhuG,Vucetic S,Brown CJ, et al. Protein flexibility and intrinsic disorder. Protein Sci 2004;13:71–80..
\section{Beyond an inventory of parts}

% NOTE: Things there wasn't space for - prediction of contacts in disordered proteins.
\section{Conclusion}
The solutions to the structures of haemoglobin and the ribosome - two iconic protein complexes, and perhaps the only recognisable in popular culture - are both considered to be, for their time, amongst the most impressive achievements in biology. It is therfore a testament to how far structural biology has progressed, that the PDB now contains over 100 haemoglobin structures, and at least fivefold more of the ribosome, all of which have been released since the turn of the century. It would be tempting to single out a particular method most responsible for this progress (certainly X-ray crystallography comes close), but in truth it has been a combination of all of the above.

In the same way as parallel improvements in electron detectors and image-processing software have had a synergistic effect in cryo-EM, so too have many other advances across the whole field of structural biology. To give a few examples: homology modelling enabled much faster processing of diffraction patterns and electron density maps, improved purification techniques enabled AP-MS, detectors designed for cryo-EM will be used with XFELs, and so on. This leaking of technologies across fields is what is driving the current rise of integrative structural biology, which seems certain to become the dominant approach over the course of the next decade. Early-career structural biologists today can no longer be content to specialise in one method or the other, but instead must be familiar with all of the topics covered in this review \cite{Shi2014,Cassiday2014}.




% NOTES below:
Conclude structural, conclude non-struc, and so on. In the structural domain, it seems likely that cryo-EM will continue to be the method of choice for protein complexes, particularly as the achievable resolutions improve. However, it is harder to predict the impact that XFELs will have due to the extraordinary leap forward they represent. There are currently several areas in which cryo-EM can make gains, but nothing like the order of magnitude improvements that could potentially be achieved by XFELs. (Is this really correct?) It might be best just to point out the density of structural biology papers in nature? How many times does cryo-EM appear in the title?

The Hein paper is very important because it shines a light on the interactome at both the level of individual complexes, but also on the broader interactions between them. Read the start of the Mann mass-spec Nature review to get a better feel for this.

Also, have ommitted a number of methods that are frequently used in order to focus on those that are most applicablew for protein complexes. For example, small angle x-ray scattering.

For non-structural methods, progress will continue as the quality of mass-spectrometers and proteomics in general catches up with that of RNA-sequencing and transcriptomics, and as researchers find more inventive ways to make use of them.

Purely computational methods are particularly attractive as they can feed off the ever increasing amounts of genomic data - methods such as EVcouplings etc. are currently limited by available sequences etc.

For the very sci-fi, quantum computers may eventually enable de novo solutions to structures, and molecular dynamics harnessed with QCs would enable some pretty mind-bending stuff.

% NOTE: This should be a really strong paragraph or two to finish off the methods review. As well as the affinity purification/XFEL example, find a few more, which together should really cement the idea that the elucidation of very large and difficult complexes is not outside our reach, and gradually the size of protein complexes will increase until we have a full understanding of not just protein complexes, but of the entire proteome, at all scales. The next step is to figure out how the layers of genomic, transcriptomic and proteomic interact with each other.
As in cryo-EM, where parallel improvements in direct electron detectors and algorithms had a synergistic effect on the field, the field of integrative structural biology should proceed along similar lines. Though I have discussed the main technologies separately in each section, progress in integrative biology as a whole should proceed faster than each component considered separately. For example, improvements in affinity purification of very low yield protein complexes can be leveraged by the reduced requirements for crystal size and quantity that is inherent in XFELs.

Wrap it up, try and find a nice way to lead into the main body of the thesis. Methods in the pipeline, anything that could be possible in theory? Assembly intermediates from cryo-EM data, use of NMR for assembly order, etc.

% NOTE: Want to try and finish strong by describing how, firstly, the methods in this chapter have enabled a unified understanding of the proteome (or are beginning too). Secondly, finish by describing how the following chapters are arranged, trying to get across the sweep through scales - focusing on an individual complex in chapter 1, implying the importance of the gain and loss of subunits leads to dramatic changes at the level of the whole organism. Next how the importance of protein complex assembly is reflected in the organisation of bacterial operons. Then how in eukaryotic cells assembly is still very important and has a major implicatuons in the observed behaviour of proteins in wild-type cells, and abnormal cells (aneuploidy).

Ultimately, the purpose of the methods described above must be to give us a complete and unified understanding of the proteome, from the level of individual complexes, to understanding the behaviour of the entire proteome in response to peturbations such as aneuploidy.

\printbibliography

\end{document}
