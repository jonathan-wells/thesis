\documentclass[a4paper,11pt,twoside,openright]{scrbook}

\usepackage{../jnwthesis}
\usepackage{lipsum}
\usepackage{standalone}
\standalonetrue

% Bibliography
% \usepackage[backend=biber,style=nature,backref=true]{biblatex}
\bibliography{/Users/jonwells/Documents/bibtex/Thesis}

% Figures
% \usepackage{graphicx}
\graphicspath{ {../figs/wip/} }

\begin{document}


\chapter{Degradation kinetics of proteins are explained by assembly of protein complexes}

\section{Introduction}

Colon cancer cells are renowned for their unusual, sometimes bizarre karyotypes. This state, in which a cell has an abnormal number of chromosomes, is known as aneuploidy, and occurs to varying degrees in all known cancers. However, it is common in most eukaryotes, regardless of whether they are capable of developing cancer. Isolates of wild yeast strains for example have been found to harbour a variety of different karyotypes, \cite{Hose2015}. In humans, approximately 0.1\% of the population carry an extra copy of chromosome 21 \cite{Presson2013}, which results in Down's syndrome. In most cases however, aneuploidies that have been acquired through the germ-line or early in development are lethal. A recent study of spontaneous miscarriages found that approximately 45\% were the result of aneuploidies \cite{Jia2015}, and the true figure is probably higher still, since studies in mice suggest that mosaic aneuploid embryos fail to develop much beyond gastrulation, and thus would often pass clinically undetected \cite{Lightfoot2006}.

\end{document}
