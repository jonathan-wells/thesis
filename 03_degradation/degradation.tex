\documentclass[a4paper,11pt,twoside,openright]{scrbook}

\usepackage{../jnwthesis}
\usepackage{lipsum}
\usepackage{mhchem}
\usepackage{standalone}
\standalonetrue

% Bibliography
% \usepackage[backend=biber,style=nature,backref=true]{biblatex}
\bibliography{/Users/jonwells/Documents/bibtex/Thesis}

% Figures
% \usepackage{graphicx}
\graphicspath{{../figs/}}

\begin{document}


\chapter{Degradation kinetics of proteins are explained by assembly of protein complexes}\label{chapter:degradation}

\section{Introduction}
In the previous chapter, we saw how gene order in bacteria is optimised for assembly of protein complexes, consistent with studies demonstrating that bacterial protein complex subunits are produced in proportions that correspond closely to the stoichiometry of the complex \cite{Li2014b,Burkhardt2017}. However, it is less clear that this is the case in eukaryotes, and the challenge they face in assembling protein complexes is compounded by the issues arising from their increased cell size and organisational complexity. Furthermore, as described in an influential paper by Papp, Pál and Hurst \cite{Papp2003}, there is a significant fitness cost to stoichiometric imbalances in the expression of eukaryotic protein complex subunits. The effects of this dosage sensitivity manifest in phenomena such as the rapid degradation of excess ribosomal subunits \cite{Warner1999,Sung2016}, reduced noise in expression of dosage sensitive genes \cite{Lehner2008} and an under-representation of heteromer subunits in regions with high copy number variation \cite{Schuster-Bockler2010} (CNV).

% NOTE: What proportion of the variance of mRNA/protein anti-correlation can be explained by degradation
How does the eukaryotic cell balance the need for rapid and efficient protein complex assembly with the pressure to avoid the deleterious consequences of imbalanced expression of subunits? There are several ways to consider this question, but one that I have found to be important comes from a study on protein degradation kinetics, carried out in collaboration with McShane et al \cite{McShane2016}. In this chapter, I will present the key results arising from this work, extended with some additional analyses relating to protein complex assembly and the balance hypothesis \cite{Papp2003}.

This project stems from a question posed by Matthias Selbach's lab, namely: does the probability of a protein molecule being degraded remain constant over its lifetime? Early studies reached the conclusion that most intracellular protein degradation follows first order kinetics, implying that proteins have a fixed probability of being degraded at any moment in time \cite{Schimke1970,Goldberg1974}. However, it is not hard to imagine scenarios in which this assumption would be violated - for example, if proteins become more unstable as they accumulate damage over time, or if they accumulate modifications marking them for destruction; alternatively, nascent proteins might initially be less stable than mature ones. Though less intuitive, there is some evidence for the latter scenario from early experiments showing that many proteins (including basigin and cystic fibrosis transmembrane conductance regulator specifically) and are degraded within the first few hours following synthesis \cite{Wheatley1980,Tyler2012,Ward1994}. Similarly it has been shown that ubiquitination of nascent proteins is common, and even occurs co-translationally with some regularity \cite{Kim2011,Wang2013}.

Considerable effort has already been invested in studying protein turnover using ribosome footprint profiling and mass spectrometric methods  \cite{Ingolia2009,Ingolia2011,Doherty2009,Schwanhausser2011,Kristensen2013}, particularly since the discordance between mRNA and protein abundance first became apparent \cite{Gygi1999a,Chen2002}. However, most of the work thus far has focussed on synthesis, since degradation is challenging to measure due to the difficulties in tracking newly synthesised proteins over extended time periods. To overcome this issue, my collaborators on the project performed pulse-chase experiments using the artificial amino acid azidohomoalanine (AHA) \cite{Kiick2002,Dieterich2006}. Combining this with SILAC mass spectrometry \cite{Ong2002a}, they were able to track protein abundance changes in mammalian cells across seven time points spanning a 32 hour period. The results from this experiment yielded exciting insights into the nature of protein complex assembly in eukaryotes, lending themselves to a simple model which successfully predicts protein behaviour in aneuploid cells.

\section{Results}
\subsection{Measuring protein degradation kinetics}
Early attempts at using pulse-chase experiments to follow protein degradation were hampered by the long pulse times required to label a sufficient proportion of proteins, which often considerably overlapped with the lifetimes of the proteins being studied. AHA is a bioorthogonal amino acid that can be incorporated by cells in place of methionine with comparatively short labelling times. In order to measure changes in labelled protein abundance over time (figure \ref{figure:pulsechase}A), mouse fibroblasts (NIH 3T3) were grown in heavy, medium and light SILAC medium, then pulse-labelled with AHA for one hour. The samples were then cold chased in AHA-free medium for durations specific to each of the three SILAC labels. Heavy cells were always harvested immediately after the pulse for use as the $t_{0}$ reference point for subsequent quantification. Medium and light samples were collected after cold chases of 1, 2, 4, 8, 16 and 32hrs, spread across three independent replicates.

\begin{figure}
    \includegraphics{c4_fig_pulsechase}
    \caption[Quantification of protein degradation kinetics by metabolic pulse-chase labelling]{\sffamily \textbf{Quantification of protein degradation kinetics by metabolic pulse-chase labelling} \\ \small (A) Experimental setup for AHA pulse chase experiment. After a 1hr AHA pulse, each SILAC culture is harvested at different time point, with heavy cells being used as the $t_{0}$ reference point. After harvesting the cell cultures are combined, AHA proteins are enriched and then quantification of abundances carried out using tandem mass spectrometry. (B) This process was repeated in triplicate to obtain seven time points at intervals from 0 to 32 hours. Plots show mass spectra for three peptides from Flna, Ctsl1 and Bsg respectively. Relative changes in abundance are calculated from the area under different peaks. (C) Processed decay profiles for 5,247 proteins, with Flna, Ctsl1 and Bsg illustrating different possible degradation profiles. This figure is adapted from figure 1, McShane et al.\cite{McShane2016}}
    \label{figure:pulsechase}
\end{figure}

Samples were combined after harvesting to avoid introducing further batch effects and AHA proteins were enriched from this mixture on an alkyne agarose resin. The relative changes in protein abundance at each time point relative to $t_{0}$ were then quantified using LC-MS/MS. Mass spectra (MS1) for peptides from Filamin  alpha (Flna), Cathespin L1 (Ctsl1) and Basigin (Bsg) are shown in figure \ref{figure:pulsechase}B. Each of these three proteins displays a different decay profile, with Flna being very stable and decaying exponentially. Ctsl1 also decays exponentially but at a much faster rate, and ceases to be detected at time points beyond 8hrs. Finally, and most interestingly, Bsg appears to decay non-exponentially, with a rapid drop in abundance over the 8hrs hours, but very little change for the last 24hrs of the experiment. This is consistent with the earlier reports of Bsg in human cell lines undergoing rapid degradation during the first few hours of its life \cite{Tyler2012}.

Controls to were carried out to verify that AHA did not induce premature translation termination or affect protein stability. Decay profiles were normalised using a set of abundant and very stable proteins to control for differences in cell number across samples. In total 5,257 proteins were quantified (figure \ref{figure:pulsechase}), and after quality control (e.g. removing proteins with fewer than 4 data points) this was reduced to 3,605 profiles suitable for further analysis. Details on these aspects of the work can be found in the methods and supplementary material of the original research paper \cite{McShane2016}.

\subsection{Many proteins are degraded non-exponentially}
Having acquired a large set of decay profiles, we next attempted to classify proteins according to their degradation profiles. Adapting a method previously used to study mRNA degradation \cite{Deneke2013}, one- and two-state Markov models were fitted to the data for each protein. Under the first model, there is a single transition probability $K_{A}$ which describes the probability of a protein being degraded at any given moment in time. Under the second, there are assumed to be two states with different degradation probabilities. A protein in state A can thus either be degraded or transition to state B, at which point its degradation probability will change. The one-state model therefore describes exponential degradation (ED), whilst the two-state model describes non-exponential degradation (NED).

\begin{figure}[h]
    \includegraphics{c4_fig_model}
    \caption[Non-exponentially degraded proteins are common]{\sffamily \textbf{Non-exponentially degraded proteins are common} \\ \small (A) One- and two-state Markov models used to model protein degradation profiles. $K_{X}$ describe transition probabilities between different states. (B) Markov models fitted to profiles of Flna, Ctsl1 and Bsg. Both Flna and Ctsl1 are equally well fitted by either model (i.e. $K_{A} \approx K_{B}$), as measured by the residual sum of squares (RSS), and therefore the simpler one-state model is selected. In contrast, Bsg is better fit by a two-state model, indicated by the favourable RSS and high AIC. (C) Under conservative thresholds for AIC and $\Delta$-score 10\% of proteins are degraded non-exponentially. Adapted from figure 2, McShane et al.\cite{McShane2016}}
    \label{figure:model}
\end{figure}

To determine which was the better fit for each protein, we used the Akaike information criterion \cite{Akaike1974} (AIC), which imposes a penalty for additional parameters and therefore gives preference to the one-state model if both explain the data similarly well. However, since AIC describes goodness of fit, but not the extent to which non-exponential degradation occurs. For example, a protein may be better fit by the two state model but only deviate weakly from a single degradation rate. To include this feature in classifications, a measure was developed ($\Delta$-score) to assess the degree to which a pre-decided time point deviates from that expected under an exponential distribution (see methods, original paper). ED proteins were then classified as those which had a 2-state $AIC < 0.2$ and $\lvert \Delta \rvert < 0.15$, NED proteins as those with $AIC > 0.8$ and $\lvert \Delta \rvert > 0.15$, with proteins not meeting either criteria being undefined. Using the example cases of Flna, Ctsl1 and Bsg (figure \ref{figure:model}B), we found that the first two are best fit by the one-state model, whereas the two-state model is preferred for Bsg, as expected from a visual assessment of the plots.

When we applied this classification scheme to the set of 3,292 proteins whose decay profiles passed quality control and for which $\Delta$-scores could be calculated, we found that the majority of proteins (49\%) were degraded exponentially, consistent with traditional models (figure \ref{figure:model}C). Strikingly however, some 10\% of proteins are best fit by a non-exponential model of degradation, despite the conservative nature of our classifier. A particularly unexpected finding was that in all NED cases, $K_{A}$ was greater than $K_{B}$, indicating that these proteins are always initially rapidly degraded, before becoming more stable later in their lives. This suggests that, at least in the 32 hour time period we used, age-related destabilisation does not occur.

\subsection{NED proteins are enriched in heteromeric protein complexes}

Based on analysis of a previously published set on manually curated protein complexes \cite{Ori2016}, my collaborators had found that NED proteins appeared to be over-represented in protein complexes. Based on this finding I mapped proteins from the degradation dataset to protein structures from the PDB, revealing that approximately 70\% of NED proteins are members of heteromeric protein complexes\footnote{Though this is probably a conservative estimate since protein complexes, particularly larger ones, are still somewhat under-sampled in the PDB relative to monomeric or homomeric structures.}, which is a significant enrichment compared to either monomers or homomers (figure \ref{figure:nedcomplex}A). To exclude the possibility that this trend was driven exclusively by the ribosome (subunits of which are prevalent in our structural dataset), I repeated this analysis with ribosomes filtered from the dataset and obtained the same trend, with comparable statistical significance (appendix figure \ref{suppfigure:ribocontrol}A).

Several independent studies have suggested that assembly of protein complexes stabilises the proteins involved \cite{Goldberg2003,Malinverni2006,Toyama2013}. These earlier studies are therefore supported by our findings that, firstly, NED proteins in our dataset exclusively transition to more stable states with age, and secondly, are enriched in heteromeric protein complexes.

% This finding was replicated using CORUM, a manually curated database of experimentally validated protein complexes (non-structural and excluding high-throughput data) compiled from the literature \cite{Ruepp2010}.

\begin{figure}
    \includegraphics{c4_fig_nedcomplex}
    \caption[Non-exponentially degraded proteins are common]{\sffamily \textbf{Non-exponentially degraded proteins are common} \\ \small (A) One- and two-state Markov models used to model protein degradation profiles. $K_{X}$ describe transition probabilities between different states. (B) Markov models fitted to profiles of Flna, Ctsl1 and Bsg. Both Flna and Ctsl1 are equally well fitted by either model (i.e. $K_{A} \approx K_{B}$), as measured by the residual sum of squares (RSS), and therefore the simpler one-state model is selected. In contrast, Bsg is better fit by a two-state model, indicated by the favourable RSS and high AIC. (C) Under conservative thresholds for AIC and $\Delta$-score 10\% of proteins are degraded non-exponentially. Adapted from figure 2, McShane et al.\cite{McShane2016}}
    \label{figure:nedcomplex}
\end{figure}

Further supporting these studies, I observed a highly significant tendency for NED proteins to be found in large heteromers (measured in terms of number of unique subunits) compared to ED proteins (figure \ref{figure:nedcomplex}B). This is a reasonable finding if one assumes that as the size of complexes increases, so too will the proportion of subunits which are protected from degradation. Again, I repeated this analysis excluding ribosomes, and the enrichment of NED proteins in large complexes remained largely unchanged (appendix figure \ref{suppfigure:ribocontrol}B).

Within individual protein complexes, NED proteins also tend to form larger interfaces than ED. Importantly, this trend holds when controlling for number of unique subunits (figure \ref{figure:nedcomplex}C), since larger complexes typically contain larger interfaces and from the last analysis we know that NED proteins are also enriched in these complexes. Intriguingly, despite the fact that NED proteins are not enriched in homomers (complexes with one unique subunit), there is nonetheless a significant tendency for homomeric NED proteins to form larger interfaces than ED.

Since there is a strong relationship between interface size and protein complex assembly order \cite{Marsh2013}, it seemed likely that NED proteins would assemble earlier. Assembly order predictions were generated for the complexes in our structural dataset, and then normalised between 0 and 1, with lower numbers indicating earlier assembly. Since the assembly order of small complexes is relatively uninformative, I restricted the analysis to protein complexes with more than five subunits. Comparing the assembly position of NED and ED subunits, there is a weak tendency for the latter to assemble later, with undefined subunits somewhere between the two (figure \ref{figure:nedcomplex}D).

Finally, I looked at the tendency for NED and ED proteins to be coexpressed with other members of the protein complex. To assign a single coexpression score to each protein complex subunit, I calculated the mean pairwise mRNA coexpression score between that subunit and all others in the complex. Comparing the two degradation classes, the expression of NED proteins is more tightly coupled to the rest of the complex (figure \ref{figure:nedcomplex}E). Since this analysis does not depend on structural information I replicated the analysis using complexes from CORUM \cite{Ruepp2010}, and obtained the same result. Similarly, repeating the analysis whilst controlling for number of unique subunits also produced a significant trend for NED proteins to be coexpressed to a greater degree with other subunits. In order to ensure that these results were not specific to the mouse cell line being used, I repeated the analyses using data generated from a human cell line (RPE1), producing highly similar results (appendix figure \ref{suppfigure:human}).

\subsection{Protein complex assembly explains degradation kinetics}
Collectively, these observations suggest that NED proteins tend to be core subunits within protein complexes, whereas ED proteins are more likely to be monomeric/homomeric or participate as peripheral subunits within heteromers. A simple model that would explain why core members of subunits would display non-exponetial degradation kinetics is shown in figure \ref{figure:nedabundance}A. If core subunits are more abundant than peripheral subunits, then there would always a fraction of proteins unable to assemble into complexes, and thus subject to rapid degradation. My collaborators estimated absolute protein abundances after the AHA pulse using iBAQ \cite{Schwanhausser2011}, and mapped these to the set of protein complexes they used previously \cite{Ori2016}. Normalising abundances relative to the mean of each complex, we found that NED proteins within complexes were significantly more abundant than ED (figure \ref{figure:nedabundance}B), consistent with our model. In addition, the second-state degradation rates of NED proteins were more similar to those of ED within complexes.

\begin{figure}[h]
\fcapsideright
    {\caption[NED proteins are produced in excess in heteromeric complexes]{\sffamily\textbf{NED proteins are produced in excess in heteromeric complexes}\newline \small Structurally, they are distinguished by their unusual tripartite ring formation, comprising two SMC arms that form a V-shaped dimer, linked by a largely disorded kleisin subunit. In eukaryotes there exist three main subfamilies of the SMC-kleisins: condensin, cohesin, and Smc5/6, whereas prokaryotes have }\label{figure:nedabundance}}
    {\includegraphics{c4_fig_abundance}}
\end{figure}

The finding that NED proteins (core protein complex subunits) tend to be over-produced relative to ED (more peripheral) is slightly paradoxical in light of the dosage balance hypothesis described by Papp et al \cite{Papp2003}. This hypothesis states that changes in gene expression that affect the stoichiometric imbalance should be deleterious, and indeed there is much evidence to support this. However, much of the evidence supporting this relates to the large changes in gene expression caused by copy number variants. However, if excess heteromeric subunits are rapidly degraded, as demonstrated by the results presented in this chapter, then this suggests that smaller fluctuations in expression might be better tolerated than fluctuations in monomeric or homomeric proteins - more specifically, increases in expression should be better tolerated than decreases. To test this idea, I mapped eQTL data from the Genotype-Tissue Expression \cite{Brown2016} (GTEx) project onto a set of human protein complexes. Separating these out by quaternary structure category revealed that, contrary to this hypothesis, the number of significant eQTLs per gene is lower for heteromeric subunits, in accordance with results supporting the balance hypothesis (figure \ref{figure:eqtls}). Moreover, there are not meaningful differences between the number of upregulating and downregulating eQTLs across different quaternary structure types.

\begin{figure}[h]
\fcapsideright
    {\caption[eQTLs are less frequent for heteromeric proteins]{\sffamily\textbf{eQTLs are less frequent for heteromeric proteins}\newline \small Structurally, they are distinguished by their unusual tripartite ring formation, comprising two SMC arms that form a V-shaped dimer, linked by a largely disorded kleisin subunit. In eukaryotes there exist three main subfamilies of the SMC-kleisins: condensin, cohesin, and Smc5/6, whereas prokaryotes have }\label{figure:eqtls}}
    {\includegraphics{c4_fig_eqtl}}
\end{figure}

\section{Discussion}

This study of protein degradation kinetics revealed that many proteins are degraded non-exponentially, indicating that examples previously reported in the literature are in fact part of a widespread phenomenon. To explain these observations, we proposed a model in which protein complex assembly stabilises those proteins involved, with excess subunits being degraded. Though I have not mentioned it in this chapter, as it will be discussed in more detail in the next, this model can be used to explain protein attenuation in aneuploid cells. However, the work also raises questions that remain as yet unanswered: what are the mechanisms that explain why assembly into complexes stabilises proteins, and why are eukaryotic protein subunits not produced at levels corresponding to their stoichiometry, as appears to be the case in bacteria \cite{Li2014}?

There are a few mechanisms could explain why proteins are stabilised upon binding. The simplest is that there is a safety-in-numbers effect, whereby proteins in large complexes that bury more surface area are less accessible to the proteasome. This is certainly consistent with the tendency of NED proteins to have larger interfaces and assemble earlier. However, this alone does not explain why NED subunits are initially degraded faster than ED subunits within complexes. One factor that is probably important is ubiquitination - it would be interesting to see if there is enrichment for ubiquitination sites in NED proteins or protein complex subunits in general, as there seems to be some evidence for \cite{Chen2014}. A question arising from observations of cotranslational ubiquitination is whether or not ubiquitinated protein are still able to form protein complexes - are ubiquitination sites enriched or depleted in protein interfaces for example?

% NOTE: Improve, tighten up and shorten a little
With respect to the second question, the published version of this article discusses a number of possible reasons why NED proteins might be overexpressed relative to ED. One interesting possibility is that it represents a simple mechanism for controlling levels of the larger complex. If the central NED subunits are constitutively overexpressed as a single, correlated group relative to late-assembling ED subunits, then the levels of the fully assembled complex can be modulated by that of the ED subunits alone. This is consistent with the observation that expression of ED subunits is often decoupled from that of the rest of the complex.

Another alternative is that overexpression of early-assembling core subunits a necessity in ensuring efficient assembly, since the encounter of interacting proteins is concentration dependent, and eukaryotes do not benefit from the high local concentration of subunits produced by encoding heteromers in operons. This scenario goes some way to explaining why eukaryotic subunits are not expressed at stoichiometric ratios, and does not contradict the balance hypothesis, since proteins may be expressed non-stoichiometrically, but nonetheless at very tightly controlled levels.

% NOTE: tighten up and finish, especially with citations.

% \printbibliography

\end{document}
